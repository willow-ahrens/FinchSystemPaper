\begin{abstract}
From FORTRAN to NumPy, arrays have revolutionized how we express computation. However, arrays in these, and almost all prominent systems, can only handle dense rectilinear integer grids.  Real world arrays often contain underlying structure, such as sparsity, runs of repeated values, or symmetry.  Support for structured data is fragmented and incomplete.  Existing frameworks limit the array structures and program control flow they support to better simplify the problem.

In this work, we propose a new programming language, Finch, which supports \textit{both} flexible control flow and diverse data structures. Finch facilitates a programming model which resolves the challenges of computing over structured arrays by combining control flow and data structures into a common representation where they can be co-optimized. Finch automatically specializes control flow to data so that performance engineers can focus on experimenting with many algorithms. Finch supports a familiar programming language of loops, statements, ifs, breaks, etc., over a wide variety of array structures, such as sparsity, run-length-encoding, symmetry, triangles, padding, or blocks. Finch reliably utilizes the key properties of structure, such as structural zeros, repeated values, or clustered non-zeros. We show that this leads to dramatic speedups in operations such as SpMV and SpGEMM, image processing, graph analytics, and a high-level tensor operator fusion interface.
\end{abstract}