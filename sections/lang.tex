
\section{The Finch Language}

\subsection{Syntax and Semantics}

$\mathbb{V}$ is the set of all values.

$\mathbb{S}$ is the set of all symbols.

$\mathbb{T}$ is the set of all types.

$I := \finchindex$

$A := \finchaccess$

$V := \finchaccess$

$E := \finchliteral | \finchvalue | \finchindex | \finchvar | \finchextent | \finchcall | \finchaccess$

$S := \finchassign | \finchloop | \finchdefine | \finchsieve | \finchblock$


% Define variables for left and right minipage widths
\newlength{\leftwidth}
\setlength{\leftwidth}{.4\linewidth}
\newlength{\rightwidth}
\setlength{\rightwidth}{.6\linewidth}

\noindent\begin{minipage}{\leftwidth}
\raggedleft $\finchliteral(val \in \mathbb{V}) :=$~
\end{minipage}%
\begin{minipage}{\rightwidth}
\begin{minted}[autogobble, linenos=false, fontsize=\normalsize]{julia}
val
\end{minted}
\end{minipage}

\noindent\begin{minipage}{\leftwidth}
\raggedleft $\finchvalue(ex \in \mathbb{S}, type \in \mathbb{T}) :=$~
\end{minipage}%
\begin{minipage}{\rightwidth}
\begin{minted}[autogobble, linenos=false, fontsize=\normalsize]{julia}
ex :: type
\end{minted}
\end{minipage}

\noindent\begin{minipage}{\leftwidth}
\raggedleft $\finchtensor(name \in \mathbb{S}) :=$~
\end{minipage}%
\begin{minipage}{\rightwidth}
\begin{minted}[autogobble, linenos=false, fontsize=\normalsize]{julia}
name
\end{minted}
\end{minipage}

\noindent\begin{minipage}{\leftwidth}
\raggedleft $\finchindex(name \in \mathbb{S}) :=$~
\end{minipage}%
\begin{minipage}{\rightwidth}
\begin{minted}[autogobble, linenos=false, fontsize=\normalsize]{julia}
name
\end{minted}
\end{minipage}

\noindent\begin{minipage}{\leftwidth}
\raggedleft $\finchvar(name \in \mathbb{S}) :=$~
\end{minipage}%
\begin{minipage}{\rightwidth}
\begin{minted}[autogobble, linenos=false, fontsize=\normalsize]{julia}
name
\end{minted}
\end{minipage}

\noindent\begin{minipage}{\leftwidth}
\raggedleft $\finchextent(a \in E, b \in E) :=$~
\end{minipage}%
\begin{minipage}{\rightwidth}
\begin{minted}[autogobble, linenos=false, fontsize=\normalsize]{julia}
a : b
\end{minted}
\end{minipage}

\noindent\begin{minipage}{\leftwidth}
\raggedleft $\finchcall(f \in E, args\ldots \in E) :=$~
\end{minipage}%
\begin{minipage}{\rightwidth}
\begin{minted}[autogobble, linenos=false, fontsize=\normalsize]{julia}
f(args...)
\end{minted}
\end{minipage}

\noindent\begin{minipage}{\leftwidth}
\raggedleft $\finchaccess(tns \in E, idxs\ldots \in E) :=$~
\end{minipage}%
\begin{minipage}{\rightwidth}
\begin{minted}[autogobble, linenos=false, fontsize=\normalsize]{julia}
tns[idxs...]
\end{minted}
\end{minipage}

\noindent\begin{minipage}{\leftwidth}
\raggedleft $\finchassign(lhs \in A, op \in E, rhs \in E) :=$~
\end{minipage}%
\begin{minipage}{\rightwidth}
\begin{minted}[autogobble, linenos=false, fontsize=\normalsize]{julia}
lhs <<op>>= rhs
\end{minted}
\end{minipage}

\noindent\begin{minipage}{\leftwidth}
\raggedleft $\finchloop(idx \in I, range \in E, body \in S) :=$~
\end{minipage}%
\begin{minipage}{\rightwidth}
\begin{minted}[autogobble, linenos=false, fontsize=\normalsize]{julia}
for idx = range; body end
\end{minted}
\end{minipage}

\noindent\begin{minipage}{\leftwidth}
\raggedleft $\finchdefine(var \in V, val \in E, body \in S) :=$~
\end{minipage}%
\begin{minipage}{\rightwidth}
\begin{minted}[autogobble, linenos=false, fontsize=\normalsize]{julia}
let var = val; body end
\end{minted}
\end{minipage}

\noindent\begin{minipage}{\leftwidth}
\raggedleft $\finchsieve(cond \in E, body \in S) :=$~
\end{minipage}%
\begin{minipage}{\rightwidth}
\begin{minted}[autogobble, linenos=false, fontsize=\normalsize]{julia}
if cond; body end
\end{minted}
\end{minipage}

\noindent\begin{minipage}{\leftwidth}
\raggedleft $\finchblock(bodies\ldots \in S) :=$~
\end{minipage}%
\begin{minipage}{\rightwidth}
\begin{minted}[autogobble, linenos=false, fontsize=\normalsize]{julia}
begin; bodies... end
\end{minted}
\end{minipage}

\subsection{Wrapper Tensors}

\subsection{Scalars}

\subsubsection{Sparse Scalars}
\subsubsection{Early Break Scalars}

