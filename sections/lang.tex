
\section{The Finch Language}

\subsection{Syntax and Semantics}

\noindent % This ensures the minipages fill the page width
\begin{minipage}{0.4\linewidth}
\begin{align*}
    \finchliteral(val \in \mathbb{V}) &:= \text{\mintinline{julia}{val}} \\
    \finchvalue(ex \in \mathbb{S}, type \in \mathbb{T}) &:= \text{\mintinline{julia}{ex :: type}} \\
    \finchtensor(name \in \mathbb{S}) &:= \text{\mintinline{julia}{name}} \\
    \finchindex(name \in \mathbb{S}) &:= \text{\mintinline{julia}{name}} \\
    \finchvar(name \in \mathbb{S}) &:= \text{\mintinline{julia}{name}} \\
    \finchextent(a \in E, b \in E) &:= \text{\mintinline{julia}{a : b}} \\
    \finchcall(f \in E, args\ldots \in E) &:= \text{\mintinline{julia}{f(args...)}} \\
    \finchaccess(tns \in T, idxs\ldots \in E) &:= \text{\mintinline{julia}{tns[idxs...]}} \\
    \finchfreeze(tns \in T) &:= \text{\mintinline{julia}{@freeze(tns)}} \\
    \finchthaw(tns \in T) &:= \text{\mintinline{julia}{@thaw(tns)}} \\
\end{align*}
\end{minipage}%
\begin{minipage}{0.5\linewidth}
\begin{align*}
    \mathbb{V} &:= \text{the set of all values.} \\
    \mathbb{S} &:= \text{the set of all symbols.} \\
    \mathbb{T} &:= \text{the set of all types.} \\
    I &:= \finchindex \\
    A &:= \finchaccess \\
    V &:= \finchvar \\
    T &:= \finchtensor \\
    E &:= \finchliteral \mid \finchvalue \mid \finchindex \\
        &\quad \mid \finchvar \mid \finchextent \mid \finchcall \mid \finchaccess \\
    S &:= \finchassign \mid \finchloop \mid \finchdefine \\
        &\quad \mid \finchsieve \mid \finchblock \\
\end{align*}
\end{minipage}%
\begin{align*}
    \finchdeclare(tns \in T, init \in E, dims\ldots \in E) &:= \text{\mintinline{julia}{tns .= init(dims...)}} \\
    \finchassign(lhs \in A, op \in E, rhs \in E) &:= \text{\mintinline{julia}{lhs \texttt{<<op>>=} rhs}} \\
    \finchloop(idx \in I, range \in E, body \in S) &:= \text{\mintinline{julia}{for idx = range; body end}} \\
    \finchdefine(var \in V, val \in E, body \in S) &:= \text{\mintinline{julia}{let var = val; body end}} \\
    \finchsieve(cond \in E, body \in S) &:= \text{\mintinline{julia}{if cond; body end}} \\
    \finchblock(bodies\ldots \in S) &:= \text{\mintinline{julia}{begin; bodies... end}}
\end{align*}


\subsection{Wrapper Tensors}

\subsection{Scalars}

\subsubsection{Sparse Scalars}
\subsubsection{Early Break Scalars}

