\section{Related Work}

The related work on array languages and libraries, both for dense and structured computation, spans several areas.
%
We highlight these works based on if they are a library or language vs. if they are dense or more general.

\paragraph{Dense Libraries:} There are many well known libraries that specialize in dense computations, exemplified by BLAS, though several BLAS routines are specialized to symmetric, hermitian, and triangular matrices~\cite{Anderson1999}.
%
This pattern is carried over into well known dense array libraries that use BLAS, exemplified by Numpy.

\paragraph{Structured Libraries:}

Many libraries support BLAS plus a few sparse array types, typically CSR, CSC, BCSR, and COO.
%
Examples include Scipy, PETSc, Armadillo, and Eigen~\cite{virtanen2020scipy, abhyankarpetsc, Rumengan2021, eigenweb}.
%
Several of these libraries also feature some graph or mesh algorithms built on sparse matrices.
%
GraphBlas exemplifies this pattern with support for many sparse matrix operations on a variety of formats~\cite{kepner2016mathematical} although a variety of other graph libraries adopt a similar approach such as LAgraph or GBase~\cite{mattson2019lagraph, kang2011gbase} among many others.



\paragraph{Dense Tensor Compilers:}

\paragraph{Tensor Compilers for Structured Data:}



%% Libraries: Numpy, LAPACK, blas, graphblas, SplAtt, Scipy, Petsc, Graph libraries (ligraph, lagraph,...)
%% Languages: Halide, Lift, TVM, OptiML
%% Languages: Lgen/Spiral, LAMapping system,
%% languages sparse: TVM Sparse, Tiachi, SPF, TACO, Stut, various extensiosn to TACO, Graphit
%% SQL: HyPer, SDQL + Formats, Vectorwise, Morphesu,

%% Most similar to us: SDQL, SPF Leader Follower, Taichi
