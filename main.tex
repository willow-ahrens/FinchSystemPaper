%%
%% This is file `sample-acmsmall.tex',
%% generated with the docstrip utility.
%%
%% The original source files were:
%%
%% samples.dtx  (with options: `all,journal,bibtex,acmsmall')
%% 
%% IMPORTANT NOTICE:
%% 
%% For the copyright see the source file.
%% 
%% Any modified versions of this file must be renamed
%% with new filenames distinct from sample-acmsmall.tex.
%% 
%% For distribution of the original source see the terms
%% for copying and modification in the file samples.dtx.
%% 
%% This generated file may be distributed as long as the
%% original source files, as listed above, are part of the
%% same distribution. (The sources need not necessarily be
%% in the same archive or directory.)
%%
%%
%% Commands for TeXCount
%TC:macro \cite [option:text,text]
%TC:macro \citep [option:text,text]
%TC:macro \citet [option:text,text]
%TC:envir table 0 1
%TC:envir table* 0 1
%TC:envir tabular [ignore] word
%TC:envir displaymath 0 word
%TC:envir math 0 word
%TC:envir comment 0 0
%%
%%
%% The first command in your LaTeX source must be the \documentclass
%% command.
%%
%% For submission and review of your manuscript please change the
%% command to \documentclass[manuscript, screen, review]{acmart}.
%%
%% When submitting camera ready or to TAPS, please change the command
%% to \documentclass[sigconf]{acmart} or whichever template is required
%% for your publication.
%%
%%
\documentclass[acmsmall]{acmart}

%%
%% \BibTeX command to typeset BibTeX logo in the docs
\AtBeginDocument{%
  \providecommand\BibTeX{{%
    Bib\TeX}}}

%% Rights management information.  This information is sent to you
%% when you complete the rights form.  These commands have SAMPLE
%% values in them; it is your responsibility as an author to replace
%% the commands and values with those provided to you when you
%% complete the rights form.
\setcopyright{acmlicensed}
\copyrightyear{2024}
\acmYear{2024}
\acmDOI{XXXXXXX.XXXXXXX}


%%
%% These commands are for a JOURNAL article.
\acmJournal{JACM}
\acmVolume{37}
\acmNumber{4}
\acmArticle{111}
\acmMonth{8}

%%
%% Submission ID.
%% Use this when submitting an article to a sponsored event. You'll
%% receive a unique submission ID from the organizers
%% of the event, and this ID should be used as the parameter to this command.
%%\acmSubmissionID{123-A56-BU3}

%%
%% For managing citations, it is recommended to use bibliography
%% files in BibTeX format.
%%
%% You can then either use BibTeX with the ACM-Reference-Format style,
%% or BibLaTeX with the acmnumeric or acmauthoryear sytles, that include
%% support for advanced citation of software artefact from the
%% biblatex-software package, also separately available on CTAN.
%%
%% Look at the sample-*-biblatex.tex files for templates showcasing
%% the biblatex styles.
%%

%%
%% The majority of ACM publications use numbered citations and
%% references.  The command \citestyle{authoryear} switches to the
%% "author year" style.
%%
%% If you are preparing content for an event
%% sponsored by ACM SIGGRAPH, you must use the "author year" style of
%% citations and references.
%% Uncommenting
%% the next command will enable that style.
%%\citestyle{acmauthoryear}

% Language setting
% Replace `english' with e.g. `spanish' to change the document language
\usepackage[english]{babel}

% Useful packages
\usepackage{amsmath}
\usepackage{graphicx}
%\usepackage{amssymb}
\usepackage{mathtools}
\usepackage{array}
\usepackage{minted}
\usepackage{ifthen}
\usepackage{ebproof}
\ebproofset{label separation=0pt}
\usepackage{multirow}


% Configuration for minted package
\setminted{
    style=colorful, % Set code style to colorful
    fontsize=\scriptsize,
    linenos, % Enable line numbers
    breaklines, % Enable line breaks
    breakanywhere, % Allow line breaks anywhere
    autogobble,
}
\setmintedinline{fontsize=\small}
%\usepackage[colorlinks=true, allcolors=blue]{hyperref}


\ifthenelse{1=1}{
\newcommand{\TODO}[1]{\textcolor{red}{TODO: #1}}
\newcommand\help[1]{\textcolor{green}{Help Needed: #1}}
\newcommand\kyle[1]{\textcolor{olive}{Kyle: #1}}
\newcommand\teo[1]{\textcolor{red}{teo: #1}}
\newcommand\willow[1]{\textcolor{purple}{Willow: #1}}
\newcommand{\saman}[1]{\textcolor{orange}{Saman: #1}}
\newcommand{\leftoff}{\textcolor{red}{\textbf{LEFT OFF}}}
\newcommand\changwan[1]{\textcolor{blue}{CW: #1}}
}{
\newcommand{\TODO}[1]{}
\newcommand{\help}[1]{}
\newcommand{\saman}[1]{}
\newcommand{\kyle}[1]{}
\newcommand{\teo}[1]{}
\newcommand{\willow}[1]{}
\newcommand{\leftoff}{}
}

\newcommand{\finchextent}{\ensuremath{\text{\textbf{extent}}}}
\newcommand{\finchliteral}{\ensuremath{\text{\textbf{literal}}}}
\newcommand{\finchvalue}{\ensuremath{\text{\textbf{value}}}}
\newcommand{\finchtensor}{\ensuremath{\text{\textbf{tensor}}}}
\newcommand{\finchmode}{\ensuremath{\text{\textbf{mode}}}}
\newcommand{\finchindex}{\ensuremath{\text{\textbf{index}}}}
\newcommand{\finchvar}{\ensuremath{\text{\textbf{variable}}}}
\newcommand{\finchcall}{\ensuremath{\text{\textbf{call}}}}
\newcommand{\finchaccess}{\ensuremath{\text{\textbf{access}}}}
\newcommand{\finchread}{\ensuremath{\text{\textbf{read}}}}
\newcommand{\finchassign}{\ensuremath{\text{\textbf{assign}}}}
\newcommand{\finchupdate}{\ensuremath{\text{\textbf{update}}}}
\newcommand{\finchloop}{\ensuremath{\text{\textbf{loop}}}}
\newcommand{\finchdefine}{\ensuremath{\text{\textbf{define}}}}
\newcommand{\finchsieve}{\ensuremath{\text{\textbf{sieve}}}}
\newcommand{\finchblock}{\ensuremath{\text{\textbf{block}}}}
\newcommand{\finchdeclare}{\ensuremath{\text{\textbf{declare}}}}
\newcommand{\finchfreeze}{\ensuremath{\text{\textbf{freeze}}}}
\newcommand{\finchthaw}{\ensuremath{\text{\textbf{thaw}}}}

\newcommand{\finchthunk}{\ensuremath{\text{\textbf{thunk}}}}
\newcommand{\finchphase}{\ensuremath{\text{\textbf{phase}}}}
\newcommand{\finchswitch}{\ensuremath{\text{\textbf{switch}}}}
\newcommand{\finchlookup}{\ensuremath{\text{\textbf{lookup}}}}
\newcommand{\finchrun}{\ensuremath{\text{\textbf{run}}}}
\newcommand{\finchspike}{\ensuremath{\text{\textbf{spike}}}}
\newcommand{\finchsequence}{\ensuremath{\text{\textbf{sequence}}}}
\newcommand{\finchstepper}{\ensuremath{\text{\textbf{stepper}}}}


\newcommand{\finchplan}{\ensuremath{\text{\textbf{plan}}}}
\newcommand{\finchquery}{\ensuremath{\text{\textbf{query}}}}
\newcommand{\finchtable}{\ensuremath{\text{\textbf{table}}}}
\newcommand{\finchalias}{\ensuremath{\text{\textbf{alias}}}}
\newcommand{\finchmapjoin}{\ensuremath{\text{\textbf{mapjoin}}}}
\newcommand{\finchaggregate}{\ensuremath{\text{\textbf{aggregate}}}}
\newcommand{\finchrelabel}{\ensuremath{\text{\textbf{relabel}}}}
\newcommand{\finchreorder}{\ensuremath{\text{\textbf{reorder}}}}
\newcommand{\finchreformat}{\ensuremath{\text{\textbf{reformat}}}}


\newcommand{\HIDE}[1]{}



%%
%% end of the preamble, start of the body of the document source.
\begin{document}


%%
%% The "title" command has an optional parameter,
%% allowing the author to define a "short title" to be used in page headers.
\title{Finch: Sparse and Structured Array Programming with Control Flow}

%%
%% The "author" command and its associated commands are used to define
%% the authors and their affiliations.
%% Of note is the shared affiliation of the first two authors, and the
%% "authornote" and "authornotemark" commands
%% used to denote shared contribution to the research.
\author{Willow Ahrens}
\affiliation{%
  \institution{MIT CSAIL}
  \city{Cambridge}
  \state{Massachusetts}
  \country{USA}}
\email{willow@csail.mit.edu}

\author{Teodoro Fields Collin}
\affiliation{%
  \institution{MIT CSAIL}
  \city{Cambridge}
  \state{Massachusetts}
  \country{USA}}
\email{teoc@mit.edu}

\author{Radha Patel}
\affiliation{%
  \institution{MIT CSAIL}
  \city{Cambridge}
  \state{Massachusetts}
  \country{USA}}
\email{rrpatel@mit.edu}

\author{Kyle Deeds}
\affiliation{%
  \institution{University of Washington}
  \city{Seattle}
  \state{Washington}
  \country{USA}}
\email{kdeeds@cs.washington.edu}

\author{Changwan Hong}
\affiliation{%
  \institution{MIT CSAIL}
  \city{Cambridge}
  \state{Massachusetts}
  \country{USA}}
\email{changwan@mit.edu}

\author{Saman Amarasinghe}
\affiliation{%
  \institution{MIT CSAIL}
  \city{Cambridge}
  \state{Massachusetts}
  \country{USA}}
\email{saman@csail.mit.edu}

%%
%% By default, the full list of authors will be used in the page
%% headers. Often, this list is too long, and will overlap
%% other information printed in the page headers. This command allows
%% the author to define a more concise list
%% of authors' names for this purpose.
\renewcommand{\shortauthors}{Ahrens et al.}

%%
%% The abstract is a short summary of the work to be presented in the
%% article.
\begin{abstract}
\HIDE{From FORTRAN to Numpy, arrays have revolutionized how we express computation.  Arrays are the highest-performing datastructure with a long history of investment and innovation, from hardware support to compiler technology.  However, arrays can only handle dense rectilinear integer grids. Real world arrays often contain underlying structure, such as sparsity, runs of repeated values, or symmetry. We describe a compiler, Finch, which adapts existing programs and interfaces to the structure and sparsity of the inputs. Finch enables programmers to capture complex, real-world data scenarios with the same productivity they expect from dense arrays. Our approach enables new loop optimizations across multiple domains, unifying techniques such as sparse tensors, databases, and lossless compression. }

Of the nearly 44 zettabytes of data humans have gathered to date, most of it is collected, processed, and stored as multi-dimensional arrays. Multi-dimensional arrays are the most prominent data type from the first programming language FORTRAN to modern Numpy. Furthermore, multi-dimensional arrays are the highest-performing data structure with a long history of investment and innovation, from hardware support to compiler
technology.  Arrays in these, and almost all prominent systems, can only handle dense rectilinear integer grids.
However, a large portion of the data, either obtained from nature or generated by humans, have structure.  

Over the years compiler researchers have valiantly attempted to put sparse tensor algebra on the same compiler transformation and code generation footing as dense tensor algebra and array codes (from Fred's thesis!). So far this has failed in both scope of the ambition and execution of comparable features.  Real world data have structure beyond sparsity including runs of repeated values, dense blocks or diagonals, ragged edges, or symmetry. 
 Thus, focusing only on sparsity limits the scope needed to handle a vast array of data \kyle{"vast array" is a good pun here, but a bit confusing to read}. Second, the arrays and loops abstraction introduced in FORTRAN supports lot more complex control flow not supported by einsum based modern sparse array compilers. For example, complex control flow such as conditional branches and early exits are needed for many important computations such as graph analytics. 
 
 In this paper, we describe a compiler, Finch,
which adapts existing programs and interfaces to the structure and sparsity of the inputs. Finch enables
programmers to capture real-world scenarios with structured data and complex control flow with the same productivity they expect from dense arrays. Our approach enables new loop optimizations across multiple domains, unifying techniques such as
sparse tensors, databases, and lossless compression.
We hope to truly bring tensor algebra on structured data to the same compiler transformation and code generation footing as dense tensor algebra and array codes.  
\end{abstract}

%%
%% The code below is generated by the tool at http://dl.acm.org/ccs.cfm.
%% Please copy and paste the code instead of the example below.
%%
\begin{CCSXML}
<ccs2012>
 <concept>
  <concept_id>00000000.0000000.0000000</concept_id>
  <concept_desc>Do Not Use This Code, Generate the Correct Terms for Your Paper</concept_desc>
  <concept_significance>500</concept_significance>
 </concept>
 <concept>
  <concept_id>00000000.00000000.00000000</concept_id>
  <concept_desc>Do Not Use This Code, Generate the Correct Terms for Your Paper</concept_desc>
  <concept_significance>300</concept_significance>
 </concept>
 <concept>
  <concept_id>00000000.00000000.00000000</concept_id>
  <concept_desc>Do Not Use This Code, Generate the Correct Terms for Your Paper</concept_desc>
  <concept_significance>100</concept_significance>
 </concept>
 <concept>
  <concept_id>00000000.00000000.00000000</concept_id>
  <concept_desc>Do Not Use This Code, Generate the Correct Terms for Your Paper</concept_desc>
  <concept_significance>100</concept_significance>
 </concept>
</ccs2012>
\end{CCSXML}

\ccsdesc[500]{Do Not Use This Code~Generate the Correct Terms for Your Paper}
\ccsdesc[300]{Do Not Use This Code~Generate the Correct Terms for Your Paper}
\ccsdesc{Do Not Use This Code~Generate the Correct Terms for Your Paper}
\ccsdesc[100]{Do Not Use This Code~Generate the Correct Terms for Your Paper}

%%
%% Keywords. The author(s) should pick words that accurately describe
%% the work being presented. Separate the keywords with commas.
\keywords{Do, Not, Us, This, Code, Put, the, Correct, Terms, for,
  Your, Paper}

\received{20 February 2007}
\received[revised]{12 March 2009}
\received[accepted]{5 June 2009}

%%
%% This command processes the author and affiliation and title
%% information and builds the first part of the formatted document.
\maketitle

\section{Introduction}

%Array programming is the core abstraction behind many of the modern miracles of computing (e.g. neural networks, scientific simulation, database processing). 
Arrays are the most fundamental abstraction in computer science. Arrays and lists are often the first-taught datastructure
\cite[Chapter 2.2]{abelson_structure_1996}, \cite[Chapter 2.2]{knuth_art_1997}.
%
Arrays are also universal across programming languages, from their introduction
in Fortran in 1957 to present-day languages like Python
\cite{backus_fortran_1957}, keeping more-or-less the same semantics.
%
Modern
array programming languages such as NumPy, SciPy, MatLab, TensorFlow, PyTorch,
and Halide have pushed the limits of productive data processing with arrays,
fueling breakthroughs in machine learning, scientific computing, image
processing, and more  \cite{harris_array_2020, virtanen_scipy_2020,
moler_history_2020, abadi_tensorflow_2016,
paszke_pytorch_2019,ragan-kelley_halide_2013}.
%
These frameworks have been the
subject of extensive industry investment to enable performant implementations,
and often operate at the peak capacity of the hardware they run on
\cite{lo_roofline_2015}.

The success and ubiquity of arrays is likely due to their simplicity. 
%
Since
their introduction, multidimensional arrays have represented dense, rectilinear,
integer grids of points. 
%
By \textbf{dense}, we mean that indices are mapped to value via a simple formula relating multidimensional space to linear memory.
%
\teo{Since the gap between the array storage and the array representation is minimal, dense arrays offer extensive compiler optimizations and extensive convenience.
%
Compilers understand dense array computation over
}
%
This
simplicity enables extensive interoperability, convenience layers, and
optimizations by breaking the abstraction barrier between array representation
and array storage.  
%
Compilers understand dense array computations across many
programming constructs, such as for and while loops, breaks, parallelism,
caching, prefetching, multiple outputs, scatters, gathers, vectoriation,
loop-carry-dependencies, and more. Several optimizations have been developed for
dense arrays, such as loop fusion, loop tiling, loop unrolling, and loop
interchange.
%
However, while dense arrays are the easiest way to program, the world is not all dense.

%Bring in structure
% - old arrays are Dense, rectilinear, integer grids
% - The world is not like that
% - Sparsity, runs of repeated values, symmetry
% - Lots of citations, perhaps a few figures from my powerpoint
%Sparse arrays
% - Networks (graphblas, graph frameworks)
% - Simulations (BLAS, symblas, banded blas, etc)
% - Databases (database engines)
%Run-length encoding
%Symmetry
%Bands
%Padding
%Blocks

\begin{figure}
	\includegraphics[width=\linewidth]{example_structures.png}
    \caption{some example structures that might be nice to put into a figure}
\end{figure}
Our world is full of structured arrays.
%
Sparse arrays (which store only nonzero elements) describe networks, databases, and simulations~\cite{abhyankarpetsc, bell2007lessons, mcauley2013hidden, balay2020petsc}.
%
Run-length encoding describes images and masks, geometry, and
databases (such as a list of transactions with the date field all the same)~\cite{shi2020column,golomb1966run}.
%
Symmetry, bands, padding, and blocks arise due to modeling choices in scientific computing (e.g. higher order FEMs) as well as in intermediate structures in many linear solvers (e.g. GMRES)~\cite{ded, saad2003iterative, o2009scientific}.
%
In the context of machine learning, combinations of sparse and blocked matrices are increasingly under consideration~\cite{dao2022monarch}.
%
Operators, such as convolution, can be expressed as structured arrays.
%
For example, a convolution with a filter can be expressed as a matrix multiplication
with the toeplitz matrix of all the circular shifts of the filter~\cite{sze2017efficient}.


%
\help{This section needs citations}
\textbf{Currently, support for structured data is fragmented and incomplete}.
%
Experts must hand write variations of even the simplest kernels, like matrix
multiply, for each data structure/data set and architecture to get performance.
%
Implementations must choose a small set of features to support well, resulting
in a compromise between \textbf{program flexibility} and \textbf{data structure
flexibility}.
%
Hand-written solutions are collected in diverse libraries like
MKL, OpenCV, LAPACK or SciPy~\cite{ bradski2000opencv, anderson1999lapack, virtanen2020scipy, psarras2022linear}. 
%
However, libraries will only ever support a subset of
programs on a subset of data structure combinations.
%
Even the most advanced
libraries, such as the GraphBLAS, which support a wide variety of sparse
operations over various semi-rings always lack support for other features, such
as tensors, fused outputs, or runs of repeated values~\cite{bulucc2017design, mattson2019lagraph}.
%
While dense array
compilers support an enormous variety of program constructs like early break and
multiple left hand sides, they only support dense arrays~\cite{ragan-kelley_halide_2013,grosser2012polly}.  
%
Special-purpose
compilers like TACO, Taichi, StructTensor, or CoRa which support a select subset of structured data
structures (only sparse, or only ragged arrays) must compromise by greatly
constraining the classes of programs which they support, such as tensor
contractions \cite{kjolstad_tensor_2019, hu_taichi_2019, ghorbani2023compiling, fegade_cora_2022}. 
%
This trade-off is visualized in Tables \ref{tab:features} and \ref{tab:data_structures}.
%


\begin{table}[h!]
  \help{This table needs citations, as well as a little bit of double checking}
  \centering
  \begin{tabular}{l|cccccc}
  \textbf{Feature / Tool} & \textbf{Halide} & \textbf{Taco} & \textbf{Cora} & \textbf{Taichi} & \textbf{StructTensor} & \textbf{Finch} \\
  \hline
  Einsums and Contractions & \checkmark & \checkmark & \checkmark & \checkmark & \checkmark & \checkmark \\
  High-Level API           &            & \checkmark &            &            &            & \checkmark \\
  Automatic API Fusion     &            &            &            &            &            & \checkmark \\
%  Parallelism             & \checkmark & \checkmark & \checkmark & \checkmark&           & \checkmark \\
  Multiple LHS             & \checkmark &            & \checkmark & \checkmark &            & \checkmark \\
  Affine Indices           & \checkmark &            &            & \checkmark & \checkmark & \checkmark \\
  Recurrence               & \checkmark &            &            &            &            &           \\
  If-Conditions and Masks  & \checkmark & \checkmark &            & \checkmark &            & \checkmark \\
  Scatter Gather           & \checkmark &            &            & \checkmark &            &\checkmark \\
  Early Break              &            & \checkmark &            & \checkmark &            &\checkmark \\
  \end{tabular}
  \caption{Feature support across various tools.}
  \label{tab:features}
  \end{table}
  
  \begin{table}[h!]
  \centering
  \begin{tabular}{l|cccccc}
  \textbf{Feature / Tool} & \textbf{Halide} & \textbf{Taco} & \textbf{Cora} & \textbf{Taichi} & \textbf{StructTensor} & \textbf{Finch} \\
  \hline
  Dense                    & \checkmark & \checkmark & \checkmark & \checkmark & \checkmark & \checkmark \\
  Padded                   & \checkmark &            &            &            &            & \checkmark \\
  One Sparse               &            & \checkmark &            & \checkmark &            &\checkmark \\
  Sparse                   &            & \checkmark &            &            &            &\checkmark \\
  Run-length               &            &            &            &            &            & \checkmark \\
  Symmetric                &            &            &            &            & \checkmark & \checkmark \\
  Regular Sparse Blocks    &            & \checkmark &            &            &            & \checkmark \\
  Irregular Sparse Blocks  &            &            &            &            &            &\checkmark \\
  Ragged                   &            &            & \checkmark &            &            & \checkmark \\
  \end{tabular}
  \caption{Support for various data structures across tools. Finch supports \textbf{both} complex programs and complex data structures.}
  \label{tab:data_structures}
\end{table}

\teo{ New paragraph:
%
Many of these special purpose compilers represent a program's control flow via a specialized representation that is tightly integrated with a narrow class of supported data structures.
%
For example, TACO's merge lattices represent iterations over subsets of rectangular integer grids where computations produce non-zero values whereas the polyhedral model allows various compilers to represent dense computations on irregular regions~\cite{kjolstad_tensor_2017, grosser2012polly}.
%
Taichi enriches single static assignment form with specialized instruction for accessing a single sparse structure drawn from a particular class, which is why it supports the most control flow for a single sparse structure~\cite{hu_taichi_2019}.
%
Furthermore, the limits and advantages of a DSL can be traced to the specific coupling of a program's control flow with the data structure: many new TACO features required modifying algorithms that worked with TACO's merge lattices and Tiramisu can express programs that Halide can't because Tiramisu uses a polyhedral model as opposed to a model built on iterating over regular grids~\cite{kjolstad_tensor_2017, henry_compilation_2021, senanayake2020sparse, baghdadi2019tiramisu}.
%
In this work, we will choose  a new representation of a program's data structures and control flows that covers a wide range of programs.
%
However, simply combining a large variety of structured data with a large variety of programs does not necessarily yield a performance advantages.
%
To find and produce efficient programs for structured data, we must overcome two challenges that occur when we intersect complex control flow with structured data:}

%before introducing this, we must discuss performance: if we want to also produce efficient code for structured data, we face two challenges.



%build on some conception of an iteration space: TACO has merge lattices, Polyhedral compilers rely on the polyhedral model, Halide and Corra uses rectangular intervals, and so on.
%
%And we believe the limitations of these systems can be traced back to their specific conceptions.} \willow{I agree with what this sentence is saying, but I'm concerned that the sentiment is unclear, because here it may seem like we say "iteration space bad" and later say "iteration space good". Can we perhaps adjust here to say may of these special purpose compilers use overly restrictive models of the iteration space of the program, but that Finch is good because it uses general control flow to express a wider variety of iteration spaces. At a high level, we need to define "iteration space" or "control flow" or "Looplets" if we want to use them in the intro, and depending on how much we want to use these words we may want to avoid them here.}
%
% \teo{For example, each modification to expressions that TACO could support required modifying the merge lattice algorithms\cite{kjolstad_tensor_2017, henry_compilation_2021, senanayake2020sparse} and several programs could not be supported in Halide, but could be in Tiramisu due to the limitations of a rectangular iteration space~\cite{baghdadi2019tiramisu}.
% %
% And many iteration space concepts simply don't allow for skipping work that isn't simply the complement of an interval.
% %
% In our work, we will show how we can use Looplets as our main tool to manage the iteration spaces of our computations and data structures.
% %
% However, to support as many data structures and iteration spaces as is implied above, we face two challenges if we want to also produce efficient code over structured data.
% }
%We see two main challenges to writing efficient code over structured data.

\textbf{Optimizations are specific to the indirection and patterns in data structures}: 
%
These structures break the simple mapping between array elements and where they are stored in memory.
%
For example, sparse arrays store lists of which coordinates are nonzero, whereas run-length-encoded arrays map several pixels to the same color value. 
%
These zero regions or repeated regions are optimization opportunities, and we must adapt the program to avoid repetitive work on these regions by referencing the stored structure.

\textbf{Performance is highly structure dependent}: Structure aware kernels are dramatically faster than unstructured ones, and the landscape of implementation decisions is dramatically unpredictable. 
%
For example, sparse kernels don't need to compute on zeros, but this means that the precise input nonzero patterns act as computational filters, affecting the runtime as they interact with each other and the implementation.


\teo{We could add a third here: making intersections worth it.}
%\teo{Data structure driven array program performance engineering?}

%% Modification: We don't propose to fix these directly, but we do propose a programming model where they can be dealt with 
%% By supporting the features above, 


\teo{Proposed new paragraph:}

\teo{We don't propose to solve these problems automatically, but we do propose a new programming language where these issues are solvable, by virtue of the precise manner in which we combine structured data and control flow.}
%
\teo{Our language, Finch, combines structured data and control flow in a predictable manner by lowering both of them into an abstraction: Looplets~\cite{ahrens_looplets_2023}, which combines structured iterators into a single control flow that only produces the relevant portions of the output.}
%
\teo{For example, if a programmer wanted to merge the even indices of two sorted arrays, a programmer could represent by point-wise multiplying two sparse vectors to produce another sparse vector and then set the even indices two zero or the programmer could only point-wise multiply the two sparse vectors under an if condition that selects even indices. (This crucial part needs improvement)}
%
\teo{Via index expressions, multiple levels of loops and tensors, and more complex filtering expressions, we can imagine many variation of this problem with fine grained distinctions about when various operations are performed.}

\teo{
Due to this fine grained mixing of control flow and data structures, we dub a new programming model for Finch: ``Fine Grained Data Structure Driven Array Programming''.
%
Finch supports \textit{both} flexible programming constructs and diverse data structures. 
%
Finch supports a similar programming language of loops, statements, if conditions, breaks, etc, over a wide variety of array structures, such as sparsity, run-length-encoding, symmetry, triangles, padding, or blocks.
%
The Finch compiler uses the structure
of the data to generate efficient implementations of these programs by predictably converting the control flow into the same representation as the data, Looplets~\cite{ahrens_looplets_2023}.}

% \begin{figure*}[t!]
%     \centering
%     \begin{subfigure}[b][0.3\textwidth]
%         \begin{minted}{julia}
%     for i = _
%         if i % 2 == 0
%             c[i] = a[i] * b[i]
%         end
%     end
%         \end{minted}
%         \caption{}
%         \label{fig:inter1}
%     \end{subfigure}
    
%     \begin{subfigure}[b][0.3\textwidth]
%         \begin{minted}{julia}
%     for i = _
%         if i % 2 == 0
%             ap[i] = a[i]
%         end
%     end
%     for i = _
%         c[i] = ap[i] * b[i]
%     end
%         \end{minted}
%         \caption{}
%         \label{fig:inter2}
%     \end{subfigure}
%     \begin{subfigure}[b][0.3\textwidth]
%         \begin{minted}{julia}
%     for i = _
%         cp[i] = ap[i] * b[i]
%     end
%     for i = _
%         if i % 2 == 0
%             c[i] = cp[i]
%         end
%     end
%         \end{minted}
%         \caption{}
%         \label{fig:inter3}
%     \end{subfigure}
%     \caption{Caption}
%     \label{fig:intersections}
% \end{figure*}

%% Many variations of this problem.
%By combining many data structures and control flows into a single language, a performance engineer could use a sufficiently predicable and fine grained compiler to find efficient programs.
%
%We dub this model, ``Datastructure-driven Array Programming'' in which the programmer specifies the structure of the data separately from the program, but then can reasonably iterate on the two with a compiler that predicable combines them, leading to efficient code eventually.
%
%(Note quite sure how much comes after this)
%We should say the predicability is generating only the non-zeros! We leverage the intiial stuff of Finch to say we will in order generate the code that only grabs the non-zeros, modified in certain ways based on the fine grained ordering of computation)
%(Use the below comment as an example - where the old paragraph talked about merging two lists.)


In this work, we propose a new programming model we call ``Datastructure-driven Array Programming'' in which the programmer specifies the structure of the data separately from the program, and the compiler uses these two descriptions to generate efficient code.\saman{Is this new? TACO did something similar?} In this model, performance engineers can more efficiently search the
complex landscape of programs and datastructures to find the best implementation. In this programming model, we can express certain concepts in programs, and others in data. For example, if we wish to merge two sorted lists, we express this as two sparse vectors which are true whenever the list contains the key in question, and we iterate over the \textit{entire} space of keys, writing to the output list whenever either of the vectors are true.
While the order in which we iterate over the data is expressed in the program, which datapoints are of interest and how to find them is expressed in the data.

We develop a new array programming language, Finch, which supports \textit{both}
flexible programming constructs and diverse data structures. Finch supports a
familiar programming language of loops, statements, if conditions, breaks, etc,
over a wide variety of array structures, such as sparsity, run-length-encoding,
symmetry, triangles, padding, or blocks. The Finch compiler uses the structure
of the data to generate efficient implementations of these programs.

\subsection{Contributions}

\begin{enumerate}
\item 
\teo{How about:
A Looplet based level-format abstraction.
%
Although many systems (TACO, Taichi, SPF, Ebb)~\cite{chou2018format,  hu_taichi_2019, strout2018sparse, bernstein2016ebb} features a flexible data structure description language for array-like computations, we show that we can describe more level formats than any other by a level format abstraction designed to target Looplets in the context of an array programming language with fine-grained control flow. 
%
The first such set of formats to efficiently capture banded,
triangular, run-length-encoded, or sparse datasets, and any combination thereof.
}

More complex array structures than ever before. A complete level-by-level
structure-description language for expressing the structure of data
hierarchically.\saman{level formats were done in TACO.} The first such set of formats to efficiently capture banded,
triangular, run-length-encoded, or sparse datasets, and any combination thereof.
\item A rich structured array programming language with for-loops
and complex control flow constructs at the same level of productivity
of dense arrays. 
%
To our knowledge, the Finch programming language is the first 
to support if-conditions, early breaks, and multiple left hand sides over
structured data, as well as complex accesses such as affine indexing or scatter/gather. 
%
\saman{for beyond dense...don't want to sound like we are the fist to do for dense. } 
\item A compiler that specializes programs to data structures 
predictably,\saman{Interesting...can we back this claim up in the discription?} facilitating an expressive language that makes it easier to search the complex space of algorithms and data structures.
%
%
\teo{
The Finch language offers extensible abstractions for elements of control flow (conditionals and indexing expressions) that the Finch compiler predictably converts into the same structured representation used to describe arrays.
%
By controlling how control flow is lowered into this representation, a programmer can control how structured data is combined with control flow to produce a single control flow, the program.
%
We observe that this strategy naturally extends the dense solution.
}
\teo{I think we can back it up in two ways:
1. We need to modify the semantics to focus when on when things are converted into Looplets
2. We need to highlight the specifics in the case studies: in particular, the blur and the symmetric operations. If we do triangle counting or something like that, we could maybe talk about that a bit.
}
\teo{
\item Extensibility of the language and cmopiler with respect to data structures and control flow: (Maybe put stuff about the data structure extensibility here?)
}

\item We evaluate the productivity of our language in several case studies,
showing that Finch can be used to accelerate a wide range of applications, 
from classic operations such as spmv and spgemm, to more complex applications such as image processing, graph analytics, and a high-level tensor operator fusion interface. \teo{Do we really want to say productivity here? What word do other similar papers use?}
%We also demonstrate how Finch can fuse high-level operations to achieve a significant speedup over non-fused kernels. Additionally, as a case study, a high-level array programming language and fusion interface for operations such as map, broadcast, or reduce that can be compiled to efficient code using the previous loop-level abstractions.
%\item A complete set of level formats for expressing data patterns hierarchically in FiberTree-style decompositions. The first such set of formats to efficiently capture banded, triangular, run-length-encoded, or sparse-run-length-encoded datasets. The formats capture many use cases, from random updates to sequential construction.
%\item The Finch array language, mirroring simple for-loops with imperative code blocks and if-conditions. The first array programming language for the above data formats to support multiple outputs, affine indexing, and imperfectly-nested loops.
%\item Tensor lifecycles, a simple constraint on tensor reads and writes that elegantly restricts Finch programs to avoid complex data dependencies, and enables tensor polymorphism by providing implementers with well-defined functions to overload.
%\item Wrapper Tensors which modify existing datastructures and recombine them to support new patterns, such as affine indexing, padding, transposition, and slicing.
%\item Wrapper Levels which modify existing datastructures and enabling complex features such as atomic updates or contiguous versus separate allocation.
%\item We define the first mappings from the existing pydata/sparse array api high-level operations to low level finch notation
%\item <Performance Contributions>
\end{enumerate}

\section{Background On Looplets}
Finch represents iteration patterns using Looplets, a language that decomposes datastructure iterators hierarchically. 
%
Looplets represent the control-flow structures needed to iterate over any given datastructure, or multiple datastructures simultaneously. 
%
In particular, looplets are good at lifting code to the highest loop level that it's needed and subdividing iteration hierarchically in coordinate space.
%
Because looplets are compiled with progressive lowering, structure-specific mathematical optimizations such as integrals, multiply by zero, etc. can be implemented using simple compiler passes like term rewriting and constant propagation during the intermediate lowering stages \cite{ahrens_looplets_2023}.

The looplets are described in Figure~\ref{fig:looplets}. We simplify the presentation to focus on the semantics, rather than precise implementation.  Several looplets introduce or modify variables in the scope of the target language. It is assumed that if a looplet introduces a variable to be used in a child looplet, the child looplet will not modify that variable.

% \begin{figure}[ht]
%     \footnotesize
%     \begin{minipage}[c]{0.65\linewidth}
%         $\finchlookup(seek, body)$: The Lookup Looplet represents a
%         randomly accessible region of an iterator. The body of the lookup is
%         understood to have one less dimension than the lookup itself, as we have
%         already ``looked up'' that index in the tensor by the time we reach the
%         body. \texttt{seek(i)} is a function that updates state to the given
%         index.
%     \end{minipage}%
%     \begin{minipage}[c]{0.35\linewidth}
%         \centering
%         \includegraphics[scale=0.20]{Looplets-lookup.png}
%     \end{minipage}
%     \vspace{3pt}

%     \begin{minipage}[c]{0.65\linewidth}
%         $\finchrun(body)$: The Run Looplet represents a constant
%         region of an iterator. The body of the run is understood to have one
%         less dimension than the lookup itself, as all of the bodies are
%         identical.
%     \end{minipage}%
%     \begin{minipage}[c]{0.35\linewidth}
%         \centering
%         \includegraphics[scale=0.20]{Looplets-run.png}
%     \end{minipage}
%     \vspace{3pt}

%     \begin{minipage}[c]{0.65\linewidth}
%         $\finchphase(c:d, body)$: The Phase Looplet represents a
%         restriction of the range on which a loop should execute, and allows us
%         to succinctly express the ranges on which children of compound looplets
%         are defined.
%     \end{minipage}%
%     \begin{minipage}[c]{0.35\linewidth}
%         \centering
%         \includegraphics[scale=0.20]{Looplets-phase.png}
%     \end{minipage}
%     \vspace{3pt}

%     \begin{minipage}[c]{0.65\linewidth}
%         $\finchswitch(cond, head, tail)$: The Switch Looplet allows
%         us to specialize the body of a looplet based on a condition, evaluated
%         in the embedding context. If the condition is true, we use `head`,
%         otherwise we use `tail`. Switch has a high lowering priority so we can
%         see what's inside of it and lower that appropriately. This also lifts
%         the condition as high as possible into the loop nest. The condition is
%         assumed to evaluate to a boolean.
%     \end{minipage}%
%     \begin{minipage}[c]{0.35\linewidth}
%         \centering
%         \includegraphics[scale=0.20]{Looplets-switch.png}
%     \end{minipage}
%     \vspace{3pt}

%     \begin{minipage}[c]{0.65\linewidth}
%         $\finchthunk(preamble, body, epilogue)$: The Thunk Looplet
%         allows us to cache certain computations in the state under which the
%         body will execute. This is useful for computing and caching the results
%         of expensive computations.
%     \end{minipage}%
%     \begin{minipage}[c]{0.35\linewidth}
%         \centering
%         \includegraphics[scale=0.20]{Looplets-thunk.png}
%     \end{minipage}
%     \vspace{3pt}

%     \begin{minipage}[c]{0.65\linewidth}
%         $\finchsequence(head, tail)$: The Sequence looplet represents the
%         concatenation of two looplets. Both arguments must be phase looplets, and
%         are assumed to be nonoverlapping, covering, and in order.
%     \end{minipage}%
%     \begin{minipage}[c]{0.35\linewidth}
%         \centering
%         \includegraphics[scale=0.20]{Looplets-sequence.png}
%     \end{minipage}
%     \vspace{3pt}

%     \begin{minipage}[c]{0.65\linewidth}
%         $\finchspike(body, tail)$ The Spike Looplet represents a run
%         followed by a single value. In this paper, Spike will be considered a
%         shorthand for $\finchsequence(\finchphase(i:j-1, \finchrun(body)),
%         \finchphase(j:j, \finchrun(tail)))$.  In the Finch compiler, spikes are
%         handled with special care, since they are an opportunity to align the
%         final run to the end of the root loop extent, without using any special
%         bounds inference.
%     \end{minipage}%
%     \begin{minipage}[c]{0.35\linewidth}
%         \centering
%         \includegraphics[scale=0.20]{Looplets-spike.png}
%     \end{minipage}
%     \vspace{3pt}

%     \begin{minipage}[c]{0.65\linewidth}
%         $\finchstepper([seek], next, body)$ The stepper looplet
%         represents a variable number of looplets, concatenated. Since our
%         looplets may be skipped over due to conditions or various rewrites, the
%         $seek$ function allows us to fast-forward the state to the start of the
%         root loop extent when it comes time to lower the stepper. The $next$
%         function advances the state to the next iteration of the stepper. 
%     \end{minipage}%
%     \begin{minipage}[c]{0.35\linewidth}
%         \centering
%         \includegraphics[scale=0.20]{Looplets-stepper.png}
%     \end{minipage}
%     \caption{The looplet language, as understood in a correct execution of a Finch program.}
%     \vspace{-8pt}
% \end{figure}


\begin{figure}[ht]
\footnotesize
\begin{tabular} {|p{0.65\linewidth}|c|} 
    \hline
    $\finchlookup(seek, body)$: The Lookup Looplet represents a randomly accessible region of an iterator. The body of the lookup is understood to have one less dimension than the lookup itself, as we have already ``looked up'' that index in the tensor by the time we reach the body. \texttt{seek(i)} is a function that updates state to the given index.
    &
    \raisebox{-\totalheight}{\includegraphics[scale=0.20]{Looplets-lookup.png}}
    \\
    \hline
    $\finchrun(body)$: The Run Looplet represents a constant region of an iterator. The body of the run is understood to have one less dimension than the lookup itself, as all of the bodies are identical.
        &
    \raisebox{-\totalheight}{\includegraphics[scale=0.20]{Looplets-run.png}}
    \\ 
    \hline
    $\finchphase(c:d, body)$: The Phase Looplet represents a restriction of the range on which a loop should execute, and allows us to succinctly express the ranges on which children of compound looplets are defined.
    &
    \raisebox{-\totalheight}{\includegraphics[scale=0.20]{Looplets-phase.png}}
    \\ 
    \hline
    $\finchswitch(cond, head, tail)$: The Switch Looplet allows us to specialize the body of a looplet based on a condition, evaluated in the embedding context. If the condition is true, we use `head`, otherwise we use `tail`. Switch has a high lowering priority so we can see what's inside of it and lower that appropriately. This also lifts the condition as high as possible into the loop nest. The condition is assumed to evaluate to a boolean.
    &
    \raisebox{-\totalheight}{\includegraphics[scale=0.20]{Looplets-switch.png}}
    \\ 
    \hline
    $\finchthunk(preamble, body, epilogue)$: The Thunk Looplet allows us to cache certain computations in the state under which the body will execute. This is useful for computing and caching the results of expensive computations.
    &
    \raisebox{-\totalheight}{\includegraphics[scale=0.20]{Looplets-thunk.png}}
    \\ 
    \hline
    $\finchsequence(head, tail)$: The Sequence looplet represents the concatenation of two looplets. Both arguments must be phase looplets, and are assumed to be nonoverlapping, covering, and in order.
    &
    \raisebox{-\totalheight}{\includegraphics[scale=0.20]{Looplets-sequence.png}}
    \\ 
    \hline
    $\finchspike(body, tail)$: The Spike Looplet represents a run followed by a single value. In this paper, Spike will be considered a shorthand for $\finchsequence(\finchphase(i:j-1, \finchrun(body)), \finchphase(j:j, \finchrun(tail)))$.  In the Finch compiler, spikes are handled with special care, since they are an opportunity to align the final run to the end of the root loop extent, without using any special bounds inference.
    &
    \raisebox{-\totalheight}{\includegraphics[scale=0.20]{Looplets-spike.png}}
    \\ 
    \hline
    $\finchstepper([seek], next, body)$: The stepper looplet represents a variable number of looplets, concatenated. Since our looplets may be skipped over due to conditions or various rewrites, the $seek$ function allows us to fast-forward the state to the start of the root loop extent when it comes time to lower the stepper. The $next$ function advances the state to the next iteration of the stepper. 
    &
    \raisebox{-\totalheight}{\includegraphics[scale=0.20]{Looplets-stepper.png}}
    \\
    \hline
    \end{tabular}
\vspace{-8pt}
\caption{The looplet language, as understood in a correct execution of a Finch program.}
\end{figure}
\section{Bridging Looplets and Finch: The Tensor Interface}

%
The Finch language provides descriptions of computations that iterate over a subset of a regular grid that is lexicographically ordered.
%
At this point, the reader might believe that compilation of a Finch program simply involves simply replacing for loops over a range with for loops over iterators, but Finch programs and data structures are sufficiently flexible that this impossible.
%
First, the Finch language interacts with multi-dimensional tensors whereas the Looplet abstraction is best suited towards iterators over a single dimension.
%
We require a bridge between the single dimensional iterators created from looplets and the mutli-dimensional abstractions common to tensor compilers.
%
Second, since the iteration order of a Finch program might not match that of a data structure (a discordant traversal), different iterators need to be requested for the same data depending on the traversal order of the program.
%
So we require a bridge that can provide different iteration orders depending on the context.
%
Third, since Finch programs can read and write to the same data, multi-dimensional tensors need to provide iterators for reading and writing as well as machinery to manage transition between these states.


To build our bridge, we embrace a set of abstractions: level formats/Fiber Trees, iteration context dependent instantiation of iterators, and tensor life cycles.
%
Our first abstraction mostly already exists in the literature: a manner of specifying a data structure for a multi-dimensional tensor out of data structures for single dimensional tensors~\cite{sze2017efficient,chou2022compilation, chou2018format}.
%
We recapitulate the essential details here.
%
Our next two abstractions add to to the first by providing a mechanism to use data structures generated by the first abstraction in a greater variety of contexts while maintaining per-dimension encapsulation of array data structures.
%
We introduce an interface to instatiate iterators in a variety of contexts in our programs and we introduce the lifecycle interface to manage when we read and write to multi-dimensional iterators.
%
These interfaces add to the level abstraction, expanding the types of data that they can express via mapping to looplets and expanding the contexts in which they can be used.
%
Previous efforts to compile a greater variety of sparse array programs left these bridges untouched ~\cite{henry_compilation_2021, won2023unified, senanayake2020sparse}.

\subsection{Level Abstraction}
Fiber-tree style tensor abstractions have been the subject of extensive study
\cite{sze2017efficient, chou2022compilation, chou2018format}.  The underlying
idea is to represent a multi-dimensional tensor as a nested vector
datastructure, where each level of the nesting corresponds to a dimension of the
tensor. Thus, a matrix would be represented as a vector of vectors. This kind of
abstraction lends itself to representing sparse tensors if we vary the type of
vector used at each level in a tree. Thus, a sparse matrix might be represented
as a dense vector of sparse vectors. The vector of subtensors in this
abstraction is referred to as a \textbf{fiber}. Prior fiber-tree representations
focus on sparsity (where only the nonzero elements are represented) and treat
sparse vectors as sets of represented points. Since our fiber-tree
represesentation must handle other kinds of structure, such as diagonal,
repeated, or constant values, we instead view each fiber as a mapping from
indices into a space of subfibers.

Instead of storing the data for each subfiber separately, most sparse tensor
formats such as CSR, DCSR, and COO usually store the data for all fibers in a
level contiguously. In this way, we can think of a level as a bulk allocator for
fibers. Continuing the analogy, we can think of each fiber as being
disambiguated by a \textbf{position}, or an index into the bulk pool of
subfibers. The mapping $f$ from indices to subfibers is thus a mapping from an
index and a position in a level to a subposition in a sublevel.
Figure~\ref{fig:levelsvsfibers} shows a simple example of a level as a pool of fibers.

When we need to refer to a particular fiber at position $p$ in the level $l$, we
may write $fiber(l, p)$. Note that the formation of fibers from levels is lazy,
and the data underlying each fiber is managed entirely by the level, so the
level may choose to overlap the storage between different fibers. Thus, the only
unique data associated with $fiber(l, p)$ is the position $p$.

\begin{figure}
    \centering
    \includegraphics[width=0.45\linewidth]{LevelsVsFibers-matrix.png}\hfill%
    \includegraphics[width=0.5\linewidth]{LevelsVsFibers-tensor.png}
    \caption{Levels and fiber tree representations of a sparse matrix and a sparse tensor. On left, a matrix is represented in a fibertree corresponding to CSC format, with a dense outer level and a sparse inner level. On right, a tensor is represented in a fibertree with two sparse outer levels, and a dense inner level. Note that the element levels in this case form the leaves of the tree.}
    \label{fig:levelsvsfibers}
\end{figure}

\subsection{Tensor Lifecycle, Declare, Freeze, Thaw, Unfurl}

Our simplified view of a level is enabled by our use of looplets to represent
the structure within each fiber. In fact, our level interface requires only
5 highly general operations, described below:

Our view of a level as a fiber allocator implies an allocation function
$assemble(tns, pos_{start}:pos_{stop})$, which allocates fibers at positions
$pos_{start}:pos_{stop}$ in the level. We don't specify a de-allocation function,
instead relying on initialization to reset the fiber if it needs to be reused.

\paragraph{$declare(lvl, init, dims...)$} Declares the level to hold subtensors of size $dims$ and an initial value of $init$. 
\paragraph{$declare(lvl, init, dims...)$} Declares the level to hold subtensors of size $dims$ and an initial value of $init$. 
\paragraph{$freeze(lvl, init, dims...)$} Declares the level to hold subtensors of size $dims$ and an initial value of $init$. 

\subsection{Core Level Language Primitives}
\begin{enumerate}
\item SparseList
\item SparseDict
\item ...
\end{enumerate}




\section{The Finch Language}

\subsection{Syntax and Semantics}

\begin{minipage}{0.6\linewidth}
    \noindent\begin{minipage}{.5\linewidth}
    \raggedleft $\finchliteral(val \in \mathbb{V}) :=$~
    \end{minipage}%
    \begin{minipage}{.5\linewidth}
    \begin{minted}[autogobble, linenos=false, fontsize=\normalsize]{julia}
    val
    \end{minted}
    \end{minipage}

    \noindent\begin{minipage}{.5\linewidth}
    \raggedleft $\finchvalue(ex \in \mathbb{S}, type \in \mathbb{T}) :=$~
    \end{minipage}%
    \begin{minipage}{.5\linewidth}
    \begin{minted}[autogobble, linenos=false, fontsize=\normalsize]{julia}
    ex :: type
    \end{minted}
    \end{minipage}
        
    \noindent\begin{minipage}{.5\linewidth}
    \raggedleft $\finchtensor(name \in \mathbb{S}) :=$~
    \end{minipage}%
    \begin{minipage}{.5\linewidth}
    \begin{minted}[autogobble, linenos=false, fontsize=\normalsize]{julia}
    name
    \end{minted}
    \end{minipage}

    \noindent\begin{minipage}{.5\linewidth}
    \raggedleft $\finchindex(name \in \mathbb{S}) :=$~
    \end{minipage}%
    \begin{minipage}{.5\linewidth}
    \begin{minted}[autogobble, linenos=false, fontsize=\normalsize]{julia}
    name
    \end{minted}
    \end{minipage}

    \noindent\begin{minipage}{.5\linewidth}
    \raggedleft $\finchvar(name \in \mathbb{S}) :=$~
    \end{minipage}%
    \begin{minipage}{.5\linewidth}
    \begin{minted}[autogobble, linenos=false, fontsize=\normalsize]{julia}
    name
    \end{minted}
    \end{minipage}

    \noindent\begin{minipage}{.5\linewidth}
    \raggedleft $\finchextent(a \in E, b \in E) :=$~
    \end{minipage}%
    \begin{minipage}{.5\linewidth}
    \begin{minted}[autogobble, linenos=false, fontsize=\normalsize]{julia}
    a : b
    \end{minted}
    \end{minipage}

    \noindent\begin{minipage}{.5\linewidth}
    \raggedleft $\finchcall(f \in E, args\ldots \in E) :=$~
    \end{minipage}%
    \begin{minipage}{.5\linewidth}
    \begin{minted}[autogobble, linenos=false, fontsize=\normalsize]{julia}
    f(args...)
    \end{minted}
    \end{minipage}
\end{minipage}

\noindent\begin{minipage}{.5\linewidth}
\raggedleft $\finchaccess(tns \in E, idxs\ldots \in E) :=$~
\end{minipage}%
\begin{minipage}{.5\linewidth}
\begin{minted}[autogobble, linenos=false, fontsize=\normalsize]{julia}
tns[idxs...]
\end{minted}
\end{minipage}

\noindent\begin{minipage}{.5\linewidth}
\raggedleft $\finchassign(lhs \in A, op \in E, rhs \in E) :=$~
\end{minipage}%
\begin{minipage}{.5\linewidth}
\begin{minted}[autogobble, linenos=false, fontsize=\normalsize]{julia}
tns[idxs...] <<op>>= rhs
\end{minted}
\end{minipage}

\noindent\begin{minipage}{.5\linewidth}
\raggedleft $\finchloop(idx \in I, range \in E, body \in S) :=$~
\end{minipage}%
\begin{minipage}{.5\linewidth}
\begin{minted}[autogobble, linenos=false, fontsize=\normalsize]{julia}
for idx = range; body end
\end{minted}
\end{minipage}

\noindent\begin{minipage}{.5\linewidth}
\raggedleft $\finchdefine(var \in V, val \in E, body \in S) :=$~
\end{minipage}%
\begin{minipage}{.5\linewidth}
\begin{minted}[autogobble, linenos=false, fontsize=\normalsize]{julia}
let var = val; body end
\end{minted}
\end{minipage}

\noindent\begin{minipage}{.5\linewidth}
\raggedleft $\finchsieve(cond \in E, body \in S) :=$~
\end{minipage}%
\begin{minipage}{.5\linewidth}
\begin{minted}[autogobble, linenos=false, fontsize=\normalsize]{julia}
if cond; body end
\end{minted}
\end{minipage}

\noindent\begin{minipage}{.5\linewidth}
\raggedleft $\finchblock(bodies\ldots \in S) :=$~
\end{minipage}%
\begin{minipage}{.5\linewidth}
\begin{minted}[autogobble, linenos=false, fontsize=\normalsize]{julia}
begin; bodies... end
\end{minted}
\end{minipage}


\subsection{Wrapper Tensors}

\subsection{Scalars}

\subsubsection{Sparse Scalars}
\subsubsection{Early Break Scalars}


\section{The Finch Compiler}

\subsection{Dimensionalization}

\subsection{Concordization}

\subsection{Bounds Analysis}

\subsection{Performance Warnings}

\subsection{Wrapperization}

\subsection{Simplification and Algebraic Transformations}

\subsection{Wrapper Tensors}

\subsection{Scalars}

\subsubsection{Sparse Scalars}
\subsubsection{Early Break Scalars}

\section{Case Studies}

We evaluate Finch on a broad set of applications to showcase it's efficiency,
flexibility, and expressibility. All of our implementations highlight the
benefits of data structure and algorithm co-design.  Our implementation of
sparse-sparse-matrix multiply (SpGEMM) translates classical lessons from sparse
performance engineering into the language of Finch, using temporaries and
randomly-accessible workspace formats to efficiently implement the three main
approaches. Our study of sparse-matrix-dense-vector multiply (SpMV) highlights
the benefits of precise structural specialization. Our studies of image
morphology and graph applications show how Finch's programming model can express more complex
real-world kernels. Finally, we explain
how flexible operators, formats, and indexing expressions in Finch have
supported a flexible implementation of the Python Array API, supporting fused exection.

All experiments were run on a single core of a 12-core 2-socket Intel Xeon E5-2695 v2 running at
2.40GHz with 128GB of memory. Finch is implemented in Julia v1.9, targeting LLVM
through Julia. All timings are the minimum of 10,000 runs or 5s of measurement,
whichever happens first.

\subsection{Sparse-Sparse Matrix Multiply (SpGEMM)}
\begin{figure}
    \begin{minipage}{0.333\linewidth}
    \begin{minted}{julia}
    @finch begin
      C .= 0
      for j=_
        for i=_
          for k=_
            C[i, j] += AT[k, i] * B[k, j]
          end
        end
      end
      return C
    end
    \end{minted}
    \end{minipage}%
    \begin{minipage}{0.333\linewidth}
    \begin{minted}{julia}
    w = Tensor(SparseByteMap(Element(0)))
    @finch begin
      C .= 0
      for j=_
        w .= 0
        for k=_
          for i=_
            w[i] += A[i, k] * B[k, j]
          end
        end
        for i=_
          C[i, j] = w[i]
        end
      end
    end
    \end{minted}
    \end{minipage}%
    \begin{minipage}{0.333\linewidth}
    \begin{minted}{julia}
    w = Tensor(SparseHash(SparseHash(Element(0))))
    @finch begin
      w .= 0
      for k=_
        for j=_
          for i=_
            w[i, j] += A[i, k] * BT[j, k]
          end
        end
      end
      C .= 0
      for j=_, i=_
        C[i, j] = w[i, j]
      end
    end
    \end{minted}
    \end{minipage}
    \vspace{-12pt}
    \caption{Inner Products, Gustavsons, and Outer Products matrix multiply in Finch}\label{fig:spgemm_listing}
    \vspace{-12pt}
\end{figure}

Sparse Matrix-Matrix Multiplication (SpGEMM) is a fundamental operation in scientific computing and data analytics. 
We compute the $M \times N$ sparse matrix $C$ as the product of $M \times K$ and $K \times N$ sparse matrices $A$ and $B$.

There are three main approaches to SpGEMM \cite[Section 2.2]{zhang2021gamma}.
%
The inner-products algorithm takes dot products of corresponding rows and columns, while the outer-products algorithm sums the outer products of corresponding columns and rows.
%
Gustavson's algorithm sums the rows of $B$ scaled by the corresponding nonzero columns in each row of $A$.
%
Inner products is known to be asymptotically less efficient than the others, as we must do a merge operation to compute each of the $O(MN)$ entries in the output \cite{ahrens2022autoscheduling}.

Figure~\ref{fig:spgemm_listing} implements all three approaches in Finch, and Figure~\ref{fig:spgemm} compares the performance of Finch to TACO.
%
Note that these algorithms mainly differ in their loop order, but that different datastructures can be used to support the various access patterns induced.
%
Although a sparse bytemap has a dense memory footprint, we use it in our Finch implemenation of Gustavson's for the smaller $O(M)$ intermediate.
%
In our Finch implementation of outer products, we use a sparse hash table, as it is fully-sparse and randomly accessible.
%
Unfortunately, TACO does not support multidimensional sparse workspaces, only supporting outer-products with a dense output, a limitation with asymptotic consequences.
%
It's also worth mentioning that the bytemap format in TACO's Gustavson's implementation is hard-wired, whereas Finch's programming model allows us to write algorithms with explicit temporary formats and transpositions.

As depicted in Figure~\ref{fig:spgemm}, Finch achieves comparable performance with TACO on smaller matrices when we use the same datastructures, and significant improvements when we use better datastructures. Finch outperforms TACO on larger matrices, with an average speedup of 1.05. \changwan{(self)may remove this number for consistency with other exps. finch\_outer\_bytemap not explained.}

\begin{figure}
	\includegraphics[width=0.5\linewidth]{spgemm_small_speedup_log_scale.png}%
	\includegraphics[width=0.5\linewidth]{spgemm_joel_speedup.png}
    \vspace{-8pt}
    \caption{A comparison of several matrix multiplication algorithms between
    Finch and Taco. On left, we use smaller matrices, ordered from small to big
    dimension. Note that inner products necessarily requires $O(n^2)$ work and
    TACO's outer products format is dense. Finch can use a sparse outer products
    format and thus has an asymptotic advantage that becomes evident as the
    output dimensions grow. On right, we use only gustavson's algorithm and
    compare on larger matrices.}
    \label{fig:spgemm}
    \vspace{-12pt}
\end{figure}

\subsection{Sparse Matrix-Vector Multiply (SpMV)}
Sparse matrix-vector multiplication (SpMV) is perhaps the most studied sparse
kernel, with a wide range of applications \cite{liu_csr5_2015,
zhou_enabling_2020}. Because SpMV is a bandwidth bound kernel, many formats have
been proposed to reduce the footprint with a minimal impact on complexity
\cite{langr_evaluation_2016}. The wide range of applications unsurprisingly
results in a wide range of structures, making it an effective kernel to
demonstrate the utility of flexible data formats. 

In this case study, we highlight different Finch formats as specified in Table \ref{spmv_tensor_formats}, and the performance effects of conforming a dataset’s structure with its storage format, which Finch's datastructure-driven model enables us to do. Finch also provides the control flow necessary to manipulate data reads and writes, enabling exploitation of multiple structural patterns concurrently (e.g. sparsity \textit{and} symmetry). 

We display speedup relative to TACO, SuiteSparseGraphBLAS, and Julia’s standard library, as depicted in Figures \ref{spmv_sorted} and \ref{spmv_grouped}.  We test using sparse matrices from a large selection of datasets spanning several previous papers: the datasets used by Vuduc et al. to test the OSKI interface \cite{vuduc2005oski}, Ahrens et al. to test a variable block row format partitioning strategy \cite{ahrens_optimal_2021}, and Kjolstad et al. to test the TACO library \cite{kjolstad_tensor_2017}. Additionally, we included the SNAP graph collection to test with boolean matrices. We also created several synthetic matrices containing bands or blocks of varying sizes as well as a permutation matrix to encapsulate a few additional use cases. The dense vector is randomly generated. We tested using the row-major and column-major Finch programs in Figure \ref{spmv_programs} as well as the symmetric program where applicable; the performance displayed for Finch on each dataset in Figure \ref{spmv_grouped} is the fastest among the formats and programs we tested. Column-major SpMV consistently performs better than row-major SpMV (an average of 1.36x better) in TACO so we use column-major SpMV in TACO as our baseline.

\begin{wrapfigure}{r}{0.41\linewidth}
    \begin{minipage}[t]{0.18\textwidth}
        \vspace{0pt} % Add this to ensure top alignment within minipage
        \begin{minted}{julia}
            y .= 0
            for j = _, i = _
              y[i] += A[i, j] * x[j]
            end
        \end{minted}
        \vspace{24pt} % Add this to ensure top alignment within minipage
        \begin{minted}{julia}
            y .= 0
            for j = _, i = _
              y[j] += A[i, j] * x[i]
            end
        \end{minted}
    \end{minipage}\hfill%
    \begin{minipage}[t]{0.22\textwidth}
        \vspace{0pt} % Add this to ensure top alignment within minipage
        \begin{minted}{julia}
            y .= 0
            for j = _
              let x_j = x[j]
                y_j .= 0
                for i = _
                  let A_ij = A[i, j]
                    y[i] += x_j * A_ij
                    y_j[] += A_ij * x[i]
                  end
                end
                #D is the diagonal
                y[j] += y_j[] + D[j] * x_j
              end
            end
        \end{minted}
    \end{minipage}
    \vspace{-8pt}
    \caption{Finch row-major, column-major and symmetric SpMV Programs}
    \label{spmv_programs}
    \vspace{-12pt}
\end{wrapfigure}



\begin{table}[htbp]
    \scriptsize
    \centering
    \vspace{-12pt}
    \caption{SpMV Tensor Formats}
    \vspace{-12pt}
    \label{spmv_tensor_formats}
    \begin{tabular}{|l|l|l|l|}
        \hline
        \textbf{Outer Level} & \textbf{Inner Level} & \textbf{Scalar Values} & \textbf{Style of Matrix}\\
        \hline
        \multirow{6}{*}{Dense} & \multirow{2}{*}{SparseList} & Element & sparse, real-valued matrices \\
        \cline{3-4} 
        & & Pattern & sparse, boolean-valued matrices \\
        \cline{2-4} 
        & SparseVBL & Element & real-valued matrices with blocked structure \\
        \cline{2-4}
        & SparseBand & Element & real-valued matrices with diagonal band \\
        \cline{2-4}
        & \multirow{2}{*}{SparsePoint} & Element & real-valued matrices with one value per row \\
        \cline{3-4} 
        & & Pattern & matrix with runs of true or false \\
        \hline 
    \end{tabular}
    \vspace{-8pt}
\end{table}

\subsubsection{Tensor Formats}
We found that the SpMV performance was superior for the level format that best paralleled the structure of the tensor. We consider the Dense(SparseList(Element)) format with the column-major SpMV program to be the Finch baseline as it is the closet analog to the sparse matrix format and SpMV program in other libraries.  

Namely, matrices with a clear blocked structure like exdata\_1, TSOPF\_RS\_b678\_c1, and heart3 performed notably well with the SparseVBL format with speedups of 2.16, 1.55, and 1.30 relative to TACO, while the baseline format had slowdowns of 0.71, 0.53, and 0.92 relative to TACO. Furthermore, the synthetic Toeplitz banded matrices we constructed performed the best with the SparseBand matrix, in particular with the toeplitz\_large\_band and the toeplitz\_medium\_band matrices having a speedup of 1.98 and 1.64 relative to TACO, while the baseline format had slowdowns of 0.84 and 0.51 relative to TACO.

There were also significant advantages of using the Pattern format instead of the Element format to represent scalar values in the matrices when these values were boolean, such as matrices in the SNAP collection which represent graph datasets are boolean. For example, the SparseList-Pattern for email-Eu-core resulted in a speedup of 2.51, while the SparseList-Element format resulted in a slowdown of 1.84 over TACO.

\begin{table}[htbp]
    \centering
    \scriptsize
    \caption{Transposed SpMV Sample Matrices}
    \vspace{-12pt}
    \label{tab:transposed_spmv_sample_matrices}
    \begin{tabular}{|
        >{\centering\arraybackslash}m{0.2\linewidth-2\tabcolsep-1.2\arrayrulewidth}|
        >{\centering\arraybackslash}m{0.2\linewidth-2\tabcolsep-1.2\arrayrulewidth}|
        >{\centering\arraybackslash}m{0.2\linewidth-2\tabcolsep-1.2\arrayrulewidth}|
        >{\centering\arraybackslash}m{0.2\linewidth-2\tabcolsep-1.2\arrayrulewidth}|
        >{\centering\arraybackslash}m{0.2\linewidth-2\tabcolsep-1.2\arrayrulewidth}|}
        \hline
         & \textbf{HB/saylr4} & \textbf{Norris/heart3} & \textbf{large\_band} & \textbf{SNAP/as-735} \\
        \hline
        \textbf{Spy} & \includegraphics[width=\linewidth]{spmv_matrices/saylr4.png} & \includegraphics[width=\linewidth]{spmv_matrices/heart3.png} & \includegraphics[width=\linewidth]{spmv_matrices/toeplitz_large_band.png} & \includegraphics[width=\linewidth]{spmv_matrices/as-735.png} \\
        \hline 
        \textbf{Group / Name} & HB/saylr4 & Norris/heart3 & large\_band & SNAP/as-735 \\
        \hline
        \textbf{Dimensions} & 3,564 x 3,564 (22,316) & 2,339 x 2,339 (680,341) & 10,000 x 10,000 (1,999,900) & 7,716 x 7,716 (26,467) \\
        \hline
        \textbf{Best Finch Format} & Symmetric SparseList & SparseVBL & SparseBand & Symmetric SparseList-Pattern \\
        \hline
    \end{tabular}
    \vspace{-8pt}
\end{table}

\subsubsection{Symmetric SpMV}
Finch enables us to exploit symmetry in the sparse matrix of the SpMV kernel by providing the capabilities to reuse memory reads and insert control flow logic to restrict iterations to either the lower or upper triangle of the sparse matrix. We can apply this strategy with any level format. Every symmetric matrix in the SparseList and SparseList-Pattern formats has better performance when we use a Finch SpMV program that takes advantage of this symmetry. However, the regular row- or column-major Finch SpMV programs have better performance for symmetric matrices than the symmetric Finch SpMV program for the other more specialized formats, likely because we need in-order accesses to fully capitalize on the specialized storage. Symmetric SpMV with the SparseList level format in Finch results in an average of 1.27x speedup over TACO and symmetric SpMV with the SparseList-Pattern format in Finch results in an average speedup of 1.21x over TACO. Notably, there is a 1.91x speedup for the HB/saylr4 matrix over TACO. 

% Perhaps add the following section again in a future iteration of paper
% \subsubsection{4D Blocked SpMV}
% Finch also provides us the capability of writing an SpMV kernel [point to figure] that computes the output in blocks. Specifically, we rewrite an n x n matrix as a n/b x n/b x b x b 4-dimensional tensor and the vector as an n x n/b matrix where b is the block-size. We represent the blocked matrix with a SparseList as its second level (i.e. of format Dense(SparseList(Dense(Dense(Element(0.0)))))) so that only the non-zero blocks are stored. Then, we perform SpMV on each b x b block individually. Note that although SparseVBL already stores consecutive nonzeros in blocks, the benefit of a 4D-blocked kernel is that it additionally computes the output block by block. This method enables us to take advantage of spatial and temporal locality via register reuse [cite].

% We evaluated the 4D-blocked kernel on the Kronecker product of a neural network matrix (erdos-renyi, with sparsity p = ?) with a blocked matrix of size 10x10 and a dense and randomly generated vector. We found a 1.04x speedup to TACO, indicating comparable performance, and a 1.35x speedup to computing a non-blocked SpMV with the SparseVBL format, the fastest performing 2D Finch format for this matrix. 

% Structured matrices with induced dense blocks of equivalent size commonly arise in machine learning applications. In feature extraction, the Kronecker product of image data and a smaller matrix is computed to extract relevant features [cite]. Graph adjacency matrices or Laplacian matrices may also exhibit dense blocks when certain subsets of nodes or edges are densely interconnected. For instance, the adjacency matrix of the Erdös-Renyi random graph has a real symmetric bxb random block at each non-vanishing entry [cite]. 


%Here's a figure with spmv_performance_sorted_(faster_than_taco).png and spmv_performance_sorted_(slower_than_taco).png

\begin{figure}
    \includegraphics[width=\linewidth]{Spmv_annotated.png}
    \vspace{-18pt}
    \caption{Performance of SpMV by Finch format.}
    \label{fig:spmv_grouped}
    \footnotesize The performance displayed for Finch on each dataset is the fastest among the formats we tested. "R" indicates row-major implementation and "C" indicates column-major implementation in Finch. The baseline Finch format is unsymmetric Dense(SparseList(Element)).
\end{figure}

\subsection{Image Morphology}

Some image processing pipelines stand to benefit from structured data processing \cite{donenfeld_unified_2022}.
In this case study, we focus on binary image morpology and the logical transformation of binary images and masks.
We consider two operations: binary erosion (computing a mask), and a masked histogram (using a mask to avoid work).

Samples from our datasets are shown in Figure~\ref{fig:image_datasets}. Note that these images are all binary, either by design or having been thresholded.

Finch allows us to modify both our algorithm and our datastructure, so we may choose to use either a dense representation with bytes ($Dense(Element(0x00))$), a bit-packed representation ($Dense(Element(UInt64))$), or a run-length encoded representation that represents runs of true or false ($SparseRLE(Pattern())$).
%
All of these have their advantages.
%
The dense representation induces the least overhead, the bit-packed representation can take advantage of bitwise binary ops, and the run-length encoded version only uses memory and compute when the pattern changes between true and false.

The erosion operation turns off a pixel unless all of it's neighbors are on.
%
This can be used to shrink the boundaries of a mask, and remove point instances of noise \cite{fisher_hypermedia_1996}.
\help{link a few cool images that show erosion}
%
Note that three instances of structured data are used here, both in the mask, in the padding of the inputs, and in the convolutional filter itself.
%
We might understand the filter as a structured tensor of circular shifts, or we might understand each shifted view of the data in an unrolled stencil computation as a structured tensor.
Figure \ref{fig:erosion_listing} displays example erosion algorithms for bitwise
or run-length-encoded algorithms, and a masked histogram kernel.

We compare to OpenCV, and chose the histogram since the OpenCV histogram function also accepts a mask. If we use $SparseRLE(Pattern())$ for our mask, we can reduce the branching
in the masked kernel and get better performance.
%
We this is also an instance where we use a computed index into the output, something not many sparse and structured programming frameworks support.

We used four datasets. We randomly selected 100 images from the MNIST \cite{lecun_gradient-based_1998} and Omniglot \cite{lake_human-level_2015} character recognition datasets, as well as a dataset of human line drawings \cite{eitz_how_2012}. We also hand-selected a subset of mask images (these images were less homogenous, so we listed them in the appendix) from a digital image processing textbook \cite{gonzalez_digital_2006}. All images were thresholded, and we also include versions of the images that have been magnified before thresholding, to induce larger constant regions. IN our erosion task, the sparseRLE format performs the best as it is asymptotically faster and uses less memory, leading to a 19.5X speedup over OpenCV on the sketches dataset, which becomes arbitrarily large as we magnify the images (here shown as 266X). The improvements with SparseRLE are seen in the histogram task as well, as it allows us to
mask off contiguous regions of computation, instead of individual pixels, reducing the branches in the mask and leading to a significant speedup (20.3x on Omniglot and 20.8x on sketches. We believe the 51.6x on MNIST is due to calling overhead in OpenCV). The bitwise kernels were effective as well, and would be more effective on datasets with less structure. A strength of Finch is that it supports structured datasets, even over bitwise operations, allowing us to implement the bitwise kernel and then mask it.

\begin{figure}
    \scriptsize
    \begin{minipage}{0.5\linewidth}
    Wordwise Erosion:
    \begin{minted}{julia}
        output .= false
        for y = _
          tmp .= false
          for x = _
            tmp[x] = coalesce(input[x, ~(y-1)], true) & input[x, y] & coalesce(input[x, ~(y+1)], true)
          end
          for x = _
            output[x, y] = coalesce(tmp[~(x-1)], true) & tmp[x] & coalesce(tmp[~(x+1)], true)
          end
        end
    \end{minted}
    \vspace{12pt}
    Masked Histogram:
    \begin{minted}{julia}
        bins .= 0 
        for x=_
          for y=_
            if mask[y, x]
              bins[div(img[y, x], 16) + 1] += 1
            end
          end
        end
    \end{minted}
    \end{minipage}%
    \begin{minipage}{0.5\linewidth}
    Bitwise Erosion:
    \begin{minted}{julia}
        output .= 0
        for y = _
          tmp .= 0
          for x = _
            if mask[x, y]
              tmp[x] = coalesce(input[x, ~(y-1)], 0xFFFFFFFF) & input[x, y] & coalesce(input[x, ~(y+1)], 0xFFFFFFFF)
            end
          end
          for x = _
            if mask[x, y]
              let tl = coalesce(tmp[~(x-1)], 0xFFFFFFFF), t = tmp[x], tr = coalesce(tmp[~(x+1)], 0xFFFFFFFF)
                let res = ((tr << (8 * sizeof(UInt) - 1)) | (t >> 1)) & t & ((t << 1) | (tl >> (8 * sizeof(UInt) - 1)))
                  output[x, y] = res
                end
              end
            end
          end
        end
    \end{minted}
    \end{minipage}
    \vspace{-12pt}
    \caption{Two approaches to erosion in Finch. The $coalesce$ function defines the out of bounds value. On left, the naive approach. On $SparseRLE(Pattern())$ inputs, this only performs operations at the boundaries of constant regions. On right, a bitwise approach, using a mask to limit work to nonzero blocks of bits.}
    \label{fig:morphology_listing}
\end{figure}

\begin{figure}
	\includegraphics[width=0.5\linewidth]{erode2_speedup_over_opencv.png}%
	\includegraphics[width=0.5\linewidth]{hist_speedup_over_opencv.png}
 \vspace{-12pt}
    \caption{Performance of Finch on image morphology tasks. On left, we run 2 iterations of erosion. On right, we run a masked histogram.}\label{fig:morphology}
\end{figure}

\subsection{Graph Analytics}
%\help{In this case, the two main highlights are that Finch can do arbitrary operators (i.e. choose), and that Finch can do early break, and also the different loop orders and multiple outputs. We may need to explain a little bit about what push pull is. For bellman, the main point is that we need multiple outputs, sparse inputs, masks, and sparse outputs with differing formats at differing points.}

We used finch to implement two fundamental graph algorithms: Breadth-First Search (BFS) and Bellman-Ford single-source shortest path. Our BFS implementation and graphs datasets are taken from Yang et. al \cite{yang_implementing_2018}, including road networks, characterized by bounded node degrees and long diameters, and scale-free graphs, where node degree distribution follows a power-law distribution and diameters are short.

Direction-optimization~\cite{beamer2012direction} is crucial for achieving high BFS performance in such scenarios, switching between push and pull traversals to efficiently explore graphs. Push traversal visits the neighbors of each frontier node, while pull traversal visits every node and checks to see if it has a neighbor in the frontier. The advantage of pull traversal is that we may terminate our search once we find a node in the frontier, saving time in the event the push traversal were to visit most of the graph anyway. Early break is the critical part of control flow in this algorithm, though the algorithms also require different loop orders, multiple outputs, and custom operators.

Figure \ref{fig:graph_result} compares performance to Graphs.jl, an open-source Julia library, and the LAGraph Library, which implements graph algorithms in the language of linear algebra using GraphBLAS\cite{mattson2019lagraph}. For the BFS algorithm, direction-optimization notably enhances performance for scale-free graphs. Although GraphBLAS uses hardwired optimizations, Finch is competitive on these benchmarks. On Bellman-Ford (with path lengths and shortest-path tree), Finch's support for multiple outputs, sparse inputs, and masks leads to superior performance over GraphBLAS (average speedup of 1.05). Note that we did not include GAP-road as it took too long to run.
 
\begin{figure}
	\includegraphics[width=0.5\linewidth]{bfs_speedup_over_graphs.jl.png}%
	\includegraphics[width=0.5\linewidth]{bellmanford_speedup_over_graphs.jl.png}
    \vspace{-12pt}
    \caption{Performance of graph apps across various tools. finch\_push\_only exclusively utilizes push traversal, while finch\_push\_pull applies direction-optimization akin to GraphBLAS. Finch's support for push/pull traversal and early break facilitates direction-optimization. Among GraphBLAS's five variants for Bellman-Ford, we selected LAGraph\_BF\_full1a, consistently the fastest with our graphs.}
     \label{fig:graph_result}
\end{figure}

\begin{figure}
    \begin{minipage}{0.33\linewidth}
    \begin{minted}{julia}
    V = Tensor(Dense(Element(false)))
    P = Tensor(Dense(Element(0)))
    F = Tensor(SparseByteMap(Pattern()))
    _F = Tensor(SparseByteMap(Pattern()))
    A = Tensor(Dense(SparseList(Pattern())))
    AT = Tensor(Dense(SparseList(Pattern())))

    function bfs_push(_F, F, A, V, P)
      @finch begin
        _F .= false
        for j=_, k=_
          if F[j] && A[k, j] && !(V[k])
            _F[k] |= true
            P[k] <<choose(0)>>= j
          end
        end
        return _F
      end
    end

    \end{minted}
\end{minipage}%
\begin{minipage}{0.33\linewidth}
    \begin{minted}{julia}
    function bfs_pull(_F, F, AT, V, P)
      p = ShortCircuitScalar{0}()
      @finch begin
        _F .= false
        for k=_
          if !V[k]
            p .= 0
            for j=_
              if F[follow(j)] && AT[j, k]
                p[] <<choose(0)>>= j
              end
            end
            if p[] != 0
              _F[k] |= true
              P[k] = p[]
            end
          end
        end
        return _F
      end
    end
    \end{minted}
\end{minipage}%
\begin{minipage}{0.33\linewidth}
  \begin{minted}{julia}
  _D = Tensor(Dense(Element(Inf)), n)
  D = Tensor(Dense(Element(Inf)), n)
  function bellmanford(A, _D, D, _F, F)
    @finch begin
    F .= false
    for j = _
      if _F[j]
        for i = _
          let d = _D[j] + A[i, j]
            D[i] <<min>>= d
            F[i] |= d < _D[i]
          end
        end
      end
    end
  end
\end{minted}
\end{minipage}
\caption{Graph Applications written in Finch. Note that parents are calculated separately for Bellman-Ford. The $choose(z)$ operator is a GraphBLAS concept which returns any argument that is not $z$.}\label{fig:graph_listing}
\end{figure}

\subsection{Implementing Numpy's Array API in Finch}
In the past decade, the adoption of the Python Array API \cite{harris_array_2020} has allowed for a proliferation array programming systems, but existing implementations of this API for structured data suffer from either incompleteness or inefficiency. They're either limited to vectors and matrices or only support tabular representations in order to reduce the complexity of interactions between different formats. Further, existing work doesn't support operator fusion which can have a drastic impact on performance as we show in Fig. \ref{fig:fusion}. We believe that a flexible compiler like Finch, which can produce efficient code for arbitrary operations between inputs in a wide variety of formats, is the secret ingredient needed to make the array API performant for structured data. To handle the expansiveness of the Array API while preserving opportunities for fusion and whole workflow optimization, we pursued a lazy evaluation strategy mediated by a high-level query language. This is implemented by 1) Finch Logic, a minimal, high-level language for expressing array operations and 2) the Finch Interpreter, which executes Finch Logic as a sequence of Finch programs.

\subsubsection{Finch Logic}
The expression fragment of the logical language takes inspiration from relational algebra while incorporating an ordering on the dimensions of a tensor. This means that operators take in a set of indexed tensors and output an indexed tensor. Conceptually, an indexed tensor can be thought of as a relation with an ordering on the index attributes and a separate value attribute, and we include the $\finchreorder$ operator to manipulate this order. To express materialization, reuse of common sub-expressions, and multiple outputs, we define $\finchquery$ and $\finchplan$. A $\finchquery$ assigns the output of an expression to a name. We use this to denote that we materialize an expression. Later queries can access the result of earlier queries through the $\finchalias$ operator, allowing multiple queries to benefit from a shared computation. A plan is a sequence of queries and outputs a set of tensors. 

\begin{align*}
    \footnotesize \finchplan(queries..., names...) \quad\quad\quad \finchquery(name, expr) \quad\quad\quad \finchreorder(expr, idxs...)\\
     \footnotesize \finchrelabel(expr, idxs...) \quad\quad\quad \finchreformat(expr, format) \quad\quad\quad \finchmapjoin(op, exprs...) \\
    \footnotesize\finchaggregate(op, expr)\quad\quad\quad\quad \finchtable(tns, idxs...)  \quad\quad\quad\quad\quad\quad \finchalias(name) \quad\quad\\
     \footnotesize expr:= \finchreorder | \finchrelabel | \finchreformat |\finchmapjoin | \finchaggregate | \finchtable | \finchalias \quad\quad
\end{align*}

Given this language, we now describe how to define a few example functions from the API as plans in the Finch Logic language. Due to the flexibility of Finch, we can use the custom operators $minby(x,y)$ (which compares $x[1]$ and $y[1]$ and returns the smaller $x$ or $y$) and $tuple(x, y)$ (which returns the tuple $(x,y)$), and we can access an index as a scalar to implement $argmin$. For conciseness, we omit the outer wrapping of $\finchplan(\finchquery(out,...),out)$.
\begin{align*}
\footnotesize &\text{sum}(M, dims=[2]) \rightarrow \finchaggregate(+,\finchrelabel(M, i_1,\ldots,i_d), i_2)\\
\footnotesize &\text{matmul}(A, B) \rightarrow \finchaggregate(+,\finchmapjoin(*, \finchrelabel(A, i, j), \finchrelabel(B, j, k)), j)\\
\footnotesize &\text{argmin}(A, dims=[2]) \rightarrow \finchaggregate(minby,\finchmapjoin(tuple, \finchrelabel(A, i_1,\ldots,i_d), \finchtable(i_2)), i_2)
\end{align*}
Notably, these plans do not specify important details about the computation such as the format of intermediates and the order of the loops. In the following discussion, we provide sensible heuristics and show that they provide good performance on important kernels. However, for larger or more complex programs, it would be important to apply a cost-based optimization strategy which we leave for future work.

\subsubsection{Standardizing \& Heuristic Optimization}
Before a plan in Finch Logic can be interpreted, it must be converted to a standard form which resolves the above questions about loop ordering and output formatting. This standard form has a few syntactic requirements, but the semantically important requirements are 1) all inputs (i.e. tables and alias operators) in a query's RHS must conform to a common ordering of the indices 2) the outermost operator of each query's RHS must be a reformat 3) the expression within the reformat must be a pointwise expression, optionally wrapped in an aggregate operator. The former allows the interpreter to identify the loop order for each kernel. The second determines the output format for each intermediate. The last one guarantees that the innermost expression can be computed as a single kernel.

During the standardization process, a concordization pass is performed which examines each query in order and selects a loop order based on a heuristic which loops over intersecting variables first. Then, each query is examined again and a transposition query is inserted for each input which doesn't match the loop order and then the input is replaced with an alias. At this point, the first semantic requirement is satisfied. Next, a formatting pass is performed which examines each query and selects a level format for each output index based on the formats of the inputs and whether random writes are required. The procedure for this is a simple set of rules which attempts to aggressively preserve the structure present in the input tensors.


\subsubsection{Finch Interpreter} 
Once the Finch Logic program has been converted to the standard form, the Finch Interpreter can execute each query, in order, through a straightforward lowering process. First, the output format is identified by unpacking the outer $\finchreformat$ statement. Next, the inner expression of the is unpacked to identify the $\finchaggregate$ operator and to convert the pointwise expression into a Finch expression. At this point, any aliases to the result of previous queries are replaced with an access to the actual tensor. Lastly, the concordant loop order is identified and instantiated. The lowered query can then be compiled and executed with the Finch compiler, and the result is assigned to $name$ in the plan's scope before proceeding to the next query in the plan. 


\subsubsection{Evaluating Finch Logic}
To demonstrate the performance of Finch Logic, we evaluate it on a series of kernels which benefit from the kind of kernel fusion that it automatically applies; 1) triangle counting on graphs 2) SDDMM 3) and element-wise operations. Further, we compare against DuckDB as a state of the art system which implements a form of kernel fusion through pipelined query execution. To do this, we express each of these kernels as a single select, join, groupby statement in SQL. For the element-wise operations, we also provide an unfused Finch implementation to show the impact of fusion. For triangle counting, we use the same set of graph matrices as in Fig. \ref{graph_result}. For SDDMM, we use this set of graph matrices for the sparse matrix, and we produce random dense matrices with embedding dimension 25. Lastly, for the elementwise operations, we use uniformly sparse matrices with dimension 10000 by 10000. A/B have sparsity $.1$, and we vary the sparsity of C in the X axis.

Across all three of these kernels, we see that the high level interface for Finch provides a significant improvement over DuckDB ranging from $1.2x-28x$. For triangle counting and SDDMM, this improvement stems from DuckDB's use of binary join plans which, while not materializing intermediates, don't optimally intersect the nonzero indices for cyclic queries. This matches with findings in the database literature showing that worst-case optimal joins (which are very similar to our kernel execution) are more efficient than binary joins for these queries \cite{wang2023free}. For the element-wise operations, this stems from Finch's ability to aggressively intersect with the indices of $C$ before handling the disjunction on the indices implied by $A+B$. DuckDB handles the latter first which makes the computation linear in the size of $A+B$ rather than $C$.

\begin{figure}
\begin{tabular}{p{0.33\textwidth} p{0.33\textwidth} p{0.33\textwidth}}
  \vspace{0pt} \includegraphics[width=140pt, height=135pt]{figures/triangle_count_speedup_over_duckdb.png} &
  \vspace{0pt} \includegraphics[width=140pt, height=135pt]{figures/sddmm_speedup_over_duckdb.png} &
  \vspace{0pt} \includegraphics[width=140pt, height=123pt]{figures/elementwise_speedup_over_duckdb.png}
\end{tabular}
\vspace{-12pt}
\label{fig:fusion}
\caption{Performance of Finch Logic for common kernels.}
\end{figure}




%matmul, mttkrp, repeated ttm, triangle counting, multiple pointwise,
%in-place.
%dot((v^t .* u), w)) vs. 
%(v^t .* dot(u, w))

\willow{Note: I may want to explain format inference somewhere in here, but I'll
have to get to it a little later, perhaps after the paper deadline}
\willow{Note: I think it would be cool to include something about how we support
numerically stable norms and argmin in this model}
%%
%% The acknowledgments section is defined using the "acks" environment
%% (and NOT an unnumbered section). This ensures the proper
%% identification of the section in the article metadata, and the
%% consistent spelling of the heading.
\begin{acks}
    To Mateusz, Hameer, and Jaeyeon for their excellent programming contributions to the Finch codebase.
\end{acks}

%%
%% The next two lines define the bibliography style to be used, and
%% the bibliography file.
\bibliographystyle{ACM-Reference-Format}
\bibliography{FinchOOPSLAWillow.bib, FinchOOPSLAOverleaf.bib}


%%
%% If your work has an appendix, this is the place to put it.
\appendix

\section Frequently Requested Finch kernels 

\end{document}
\endinput