%%
%% This is file `sample-acmsmall.tex',
%% generated with the docstrip utility.
%%
%% The original source files were:
%%
%% samples.dtx  (with options: `all,journal,bibtex,acmsmall')
%% 
%% IMPORTANT NOTICE:
%% 
%% For the copyright see the source file.
%% 
%% Any modified versions of this file must be renamed
%% with new filenames distinct from sample-acmsmall.tex.
%% 
%% For distribution of the original source see the terms
%% for copying and modification in the file samples.dtx.
%% 
%% This generated file may be distributed as long as the
%% original source files, as listed above, are part of the
%% same distribution. (The sources need not necessarily be
%% in the same archive or directory.)
%%
%%
%% Commands for TeXCount
%TC:macro \cite [option:text,text]
%TC:macro \citep [option:text,text]
%TC:macro \citet [option:text,text]
%TC:envir table 0 1
%TC:envir table* 0 1
%TC:envir tabular [ignore] word
%TC:envir displaymath 0 word
%TC:envir math 0 word
%TC:envir comment 0 0
%%
%%
%% The first command in your LaTeX source must be the \documentclass
%% command.
%%
%% For submission and review of your manuscript please change the
%% command to \documentclass[manuscript, screen, review]{acmart}.
%%
%% When submitting camera ready or to TAPS, please change the command
%% to \documentclass[sigconf]{acmart} or whichever template is required
%% for your publication.
%%
%%
\documentclass[acmsmall]{acmart}

%%
%% \BibTeX command to typeset BibTeX logo in the docs
\AtBeginDocument{%
  \providecommand\BibTeX{{%
    Bib\TeX}}}

%% Rights management information.  This information is sent to you
%% when you complete the rights form.  These commands have SAMPLE
%% values in them; it is your responsibility as an author to replace
%% the commands and values with those provided to you when you
%% complete the rights form.
\setcopyright{acmlicensed}
\copyrightyear{2024}
\acmYear{2024}
\acmDOI{XXXXXXX.XXXXXXX}


%%
%% These commands are for a JOURNAL article.
\acmJournal{JACM}
\acmVolume{37}
\acmNumber{4}
\acmArticle{111}
\acmMonth{8}

%%
%% Submission ID.
%% Use this when submitting an article to a sponsored event. You'll
%% receive a unique submission ID from the organizers
%% of the event, and this ID should be used as the parameter to this command.
%%\acmSubmissionID{123-A56-BU3}

%%
%% For managing citations, it is recommended to use bibliography
%% files in BibTeX format.
%%
%% You can then either use BibTeX with the ACM-Reference-Format style,
%% or BibLaTeX with the acmnumeric or acmauthoryear sytles, that include
%% support for advanced citation of software artefact from the
%% biblatex-software package, also separately available on CTAN.
%%
%% Look at the sample-*-biblatex.tex files for templates showcasing
%% the biblatex styles.
%%

%%
%% The majority of ACM publications use numbered citations and
%% references.  The command \citestyle{authoryear} switches to the
%% "author year" style.
%%
%% If you are preparing content for an event
%% sponsored by ACM SIGGRAPH, you must use the "author year" style of
%% citations and references.
%% Uncommenting
%% the next command will enable that style.
%%\citestyle{acmauthoryear}

% Language setting
% Replace `english' with e.g. `spanish' to change the document language
\usepackage[english]{babel}

% Useful packages
\usepackage{amsmath}
\usepackage{graphicx}
%\usepackage{amssymb}
\usepackage{array}
%\usepackage[colorlinks=true, allcolors=blue]{hyperref}

\newcommand\teo[1]{\textcolor{red}{teo:#1}}

%%
%% end of the preamble, start of the body of the document source.
\begin{document}


%%
%% The "title" command has an optional parameter,
%% allowing the author to define a "short title" to be used in page headers.
\title{Finch: A Datastructure-Driven Array Programming Language}

%%
%% The "author" command and its associated commands are used to define
%% the authors and their affiliations.
%% Of note is the shared affiliation of the first two authors, and the
%% "authornote" and "authornotemark" commands
%% used to denote shared contribution to the research.
\author{Willow Ahrens}
\affiliation{%
  \institution{MIT CSAIL}
  \city{Cambridge}
  \state{Massachusetts}
  \country{USA}}
\email{willow@csail.mit.edu}

\author{Teo Collins}
\affiliation{%
  \institution{MIT CSAIL}
  \city{Cambridge}
  \state{Massachusetts}
  \country{USA}}
\email{teoc@mit.edu}

\author{Radha Patel}
\affiliation{%
  \institution{MIT CSAIL}
  \city{Cambridge}
  \state{Massachusetts}
  \country{USA}}
\email{rrpatel@mit.edu}

\author{Kyle Deeds}
\affiliation{%
  \institution{University of Washington}
  \city{Seattle}
  \state{Washington}
  \country{USA}}
\email{kdeeds@cs.washington.edu}

\author{Changwan Hong}
\affiliation{%
  \institution{MIT CSAIL}
  \city{Cambridge}
  \state{Massachusetts}
  \country{USA}}
\email{changwan@mit.edu}

\author{Saman Amarasinghe}
\affiliation{%
  \institution{MIT CSAIL}
  \city{Cambridge}
  \state{Massachusetts}
  \country{USA}}
\email{saman@csail.mit.edu}

%%
%% By default, the full list of authors will be used in the page
%% headers. Often, this list is too long, and will overlap
%% other information printed in the page headers. This command allows
%% the author to define a more concise list
%% of authors' names for this purpose.
\renewcommand{\shortauthors}{Ahrens et al.}

%%
%% The abstract is a short summary of the work to be presented in the
%% article.
\begin{abstract}
From FORTRAN to Numpy, arrays have revolutionized how we express computation.  Arrays are the highest-performing datastructure with a long history of investment and innovation, from hardware support to compiler technology.  However, arrays can only handle dense rectilinear integer grids. Real world arrays often contain underlying structure, such as sparsity, runs of repeated values, or symmetry. We describe a compiler, Finch, which adapts existing programs and interfaces to the structure and sparsity of the inputs. Finch enables programmers to capture complex, real-world data scenarios with the same productivity they expect from dense arrays. Our approach enables new loop optimizations across multiple domains, unifying techniques such as sparse tensors, databases, and lossless compression. 
\end{abstract}

%%
%% The code below is generated by the tool at http://dl.acm.org/ccs.cfm.
%% Please copy and paste the code instead of the example below.
%%
\begin{CCSXML}
<ccs2012>
 <concept>
  <concept_id>00000000.0000000.0000000</concept_id>
  <concept_desc>Do Not Use This Code, Generate the Correct Terms for Your Paper</concept_desc>
  <concept_significance>500</concept_significance>
 </concept>
 <concept>
  <concept_id>00000000.00000000.00000000</concept_id>
  <concept_desc>Do Not Use This Code, Generate the Correct Terms for Your Paper</concept_desc>
  <concept_significance>300</concept_significance>
 </concept>
 <concept>
  <concept_id>00000000.00000000.00000000</concept_id>
  <concept_desc>Do Not Use This Code, Generate the Correct Terms for Your Paper</concept_desc>
  <concept_significance>100</concept_significance>
 </concept>
 <concept>
  <concept_id>00000000.00000000.00000000</concept_id>
  <concept_desc>Do Not Use This Code, Generate the Correct Terms for Your Paper</concept_desc>
  <concept_significance>100</concept_significance>
 </concept>
</ccs2012>
\end{CCSXML}

\ccsdesc[500]{Do Not Use This Code~Generate the Correct Terms for Your Paper}
\ccsdesc[300]{Do Not Use This Code~Generate the Correct Terms for Your Paper}
\ccsdesc{Do Not Use This Code~Generate the Correct Terms for Your Paper}
\ccsdesc[100]{Do Not Use This Code~Generate the Correct Terms for Your Paper}

%%
%% Keywords. The author(s) should pick words that accurately describe
%% the work being presented. Separate the keywords with commas.
\keywords{Do, Not, Us, This, Code, Put, the, Correct, Terms, for,
  Your, Paper}

\received{20 February 2007}
\received[revised]{12 March 2009}
\received[accepted]{5 June 2009}

%%
%% This command processes the author and affiliation and title
%% information and builds the first part of the formatted document.
\maketitle

\section{Introduction}

%- Fortran supported lots of control structures, no data structures, just dense arrays.
%
%- We tried to emulate complex data using only dense arrays but with complicated control structures.
%
%- Recently, people tried to build frameworks to support structured data, but gave up a lot of program side
%
%- We’re bringing these together, both need to work together.
\subsection{Contributions}

\begin{enumerate}
\item A rich structured array programming language with for-loops
and complex control flow constructs at the same level of productivity
of dense arrays. To our knowledge, the Finch programming language is the first
to support if-conditions, early breaks, and multiple left hand sides over
structured data, as well as complex accesses such as affine indexing or scatter/gather.
\item More complex array structures than ever before. A complete level-by-level
structure-description language for expressing the structure of data
hierarchically. The first such set of formats to efficiently capture banded,
triangular, run-length-encoded, or sparse datasets, and any combination thereof.
\item The Finch compiler specializes programs to data structures in a
predictable, deterministic approach, making it easier to search the complex
space of programs and datastructures to find an appropriate fit for a given
application. The tensor interface makes it easy to extend Finch to new level
formats. A unique tensor lifecycle model enables polymorphism by analyzing the
appropriate stages to insert simple, overloadable, interface functions such as
initialization or finalization. 

\item We evaluate the productivity of our language in several case studies,
showing that Finch can be used to accelerate a wide range of applications, 
from classic operations such as spmv and spgemm, to more complex applications such as image processing and graph analytics.
We also demonstrate how Finch can fuse high-level operations to achieve a significant speedup over non-fused kernels. Additionally, as a case study, a high-level array programming language and fusion interface for
operations such as map, broadcast, or reduce that can be compiled to efficient
code using the previous loop-level abstractions.
%\item A complete set of level formats for expressing data patterns hierarchically in FiberTree-style decompositions. The first such set of formats to efficiently capture banded, triangular, run-length-encoded, or sparse-run-length-encoded datasets. The formats capture many use cases, from random updates to sequential construction.
%\item The Finch array language, mirroring simple for-loops with imperative code blocks and if-conditions. The first array programming language for the above data formats to support multiple outputs, affine indexing, and imperfectly-nested loops.
%\item Tensor lifecycles, a simple constraint on tensor reads and writes that elegantly restricts Finch programs to avoid complex data dependencies, and enables tensor polymorphism by providing implementers with well-defined functions to overload.
%\item Wrapper Tensors which modify existing datastructures and recombine them to support new patterns, such as affine indexing, padding, transposition, and slicing.
%\item Wrapper Levels which modify existing datastructures and enabling complex features such as atomic updates or contiguous versus separate allocation.
%\item We define the first mappings from the existing pydata/sparse array api high-level operations to low level finch notation
%\item <Performance Contributions>
\end{enumerate}

\begin{table}[h!]
\centering
\begin{tabular}{l|ccccc}
\textbf{Feature / Tool} & \textbf{Halide} & \textbf{Taco} & \textbf{Cora} & \textbf{Taichi} & \textbf{Finch} \\
\hline
Einsums and Contractions & \checkmark & \checkmark & \checkmark & \checkmark & \checkmark \\
Parallelism             & \checkmark & \checkmark & \checkmark & \checkmark & \checkmark \\
Multiple LHS            & \checkmark &            & \checkmark & \checkmark & \checkmark \\
Affine Indices          & \checkmark &            &            & \checkmark & \checkmark \\
Recurrence              & \checkmark &            &            &            &            \\
If-Conditions and Masks & \checkmark & \checkmark &            & \checkmark & \checkmark \\
Scatter Gather          & \checkmark &            &            & \checkmark & \checkmark \\
Early Break             &            & \checkmark &            &   \checkmark         & \checkmark \\
\end{tabular}
\caption{Feature support across various tools.}
\label{tab:features}
\end{table}

\begin{table}[h!]
\centering
\begin{tabular}{l|ccccc}
\textbf{Feature / Tool} & \textbf{Halide} & \textbf{Taco} & \textbf{Cora} & \textbf{Taichi} & \textbf{Finch} \\
\hline
Dense                    & \checkmark & \checkmark & \checkmark & \checkmark & \checkmark \\
Padded                   & \checkmark &            &            &            & \checkmark \\
One Sparse               &            & \checkmark &            & \checkmark & \checkmark \\
Sparse                   &            & \checkmark &            &            & \checkmark \\
Run-length               &            &            &            &            & \checkmark \\
Symmetric                &            &            &            &            & \checkmark \\
Regular Sparse Blocks    &            & \checkmark &            &            & \checkmark \\
Irregular Sparse Blocks  &            &            &            &            & \checkmark \\
Ragged                   &            &            & \checkmark &            & \checkmark \\
\end{tabular}
\caption{Support for various data structures across tools.}
\label{tab:data_structures}
\end{table}

\section{Background}
\subsection{Looplets}
\subsection{FiberTrees}

\subsection{Concordant Iteration}

\subsection{Protocols}

\section{The Finch Language}

\subsection{Syntax and Semantics}


\section{Bridging Looplets and Finch: The Tensor Interface}
\teo{I am doing stuff in this section.}
\section{Bridging Looplets and Finch: The Tensor Interface}

%
The Finch language provides descriptions of computations that iterate over a subset of a regular grid that is lexicographically ordered.
%
At this point, the reader might believe that compilation of a Finch program simply involves simply replacing for loops over a range with for loops over iterators, but Finch programs and data structures are sufficiently flexible that this impossible.
%
First, the Finch language interacts with multi-dimensional tensors whereas the Looplet abstraction is best suited towards iterators over a single dimension.
%
We require a bridge between the single dimensional iterators created from looplets and the mutli-dimensional abstractions common to tensor compilers.
%
Second, since the iteration order of a Finch program might not match that of a data structure (a discordant traversal), different iterators need to be requested for the same data depending on the traversal order of the program.
%
So we require a bridge that can provide different iteration orders depending on the context.
%
Third, since Finch programs can read and write to the same data, multi-dimensional tensors need to provide iterators for reading and writing as well as machinery to manage transition between these states.


To build our bridge, we embrace a set of abstractions: level formats/Fiber Trees, iteration context dependent instantiation of iterators, and tensor life cycles.
%
Our first abstraction mostly already exists in the literature: a manner of specifying a data structure for a multi-dimensional tensor out of data structures for single dimensional tensors~\cite{sze2017efficient,chou2022compilation, chou2018format}.
%
We recapitulate the essential details here.
%
Our next two abstractions add to to the first by providing a mechanism to use data structures generated by the first abstraction in a greater variety of contexts while maintaining per-dimension encapsulation of array data structures.
%
We introduce an interface to instatiate iterators in a variety of contexts in our programs and we introduce the lifecycle interface to manage when we read and write to multi-dimensional iterators.
%
These interfaces add to the level abstraction, expanding the types of data that they can express via mapping to looplets and expanding the contexts in which they can be used.
%
Previous efforts to compile a greater variety of sparse array programs left these bridges untouched ~\cite{henry_compilation_2021, won2023unified, senanayake2020sparse}.

\subsection{Level Abstraction}
Fiber-tree style tensor abstractions have been the subject of extensive study
\cite{sze2017efficient, chou2022compilation, chou2018format}.  The underlying
idea is to represent a multi-dimensional tensor as a nested vector
datastructure, where each level of the nesting corresponds to a dimension of the
tensor. Thus, a matrix would be represented as a vector of vectors. This kind of
abstraction lends itself to representing sparse tensors if we vary the type of
vector used at each level in a tree. Thus, a sparse matrix might be represented
as a dense vector of sparse vectors. The vector of subtensors in this
abstraction is referred to as a \textbf{fiber}. Prior fiber-tree representations
focus on sparsity (where only the nonzero elements are represented) and treat
sparse vectors as sets of represented points. Since our fiber-tree
represesentation must handle other kinds of structure, such as diagonal,
repeated, or constant values, we instead view each fiber as a mapping from
indices into a space of subfibers.

Instead of storing the data for each subfiber separately, most sparse tensor
formats such as CSR, DCSR, and COO usually store the data for all fibers in a
level contiguously. In this way, we can think of a level as a bulk allocator for
fibers. Continuing the analogy, we can think of each fiber as being
disambiguated by a \textbf{position}, or an index into the bulk pool of
subfibers. The mapping $f$ from indices to subfibers is thus a mapping from an
index and a position in a level to a subposition in a sublevel.
Figure~\ref{fig:levelsvsfibers} shows a simple example of a level as a pool of fibers.

When we need to refer to a particular fiber at position $p$ in the level $l$, we
may write $fiber(l, p)$. Note that the formation of fibers from levels is lazy,
and the data underlying each fiber is managed entirely by the level, so the
level may choose to overlap the storage between different fibers. Thus, the only
unique data associated with $fiber(l, p)$ is the position $p$.

\begin{figure}
    \centering
    \includegraphics[width=0.45\linewidth]{LevelsVsFibers-matrix.png}\hfill%
    \includegraphics[width=0.5\linewidth]{LevelsVsFibers-tensor.png}
    \caption{Levels and fiber tree representations of a sparse matrix and a sparse tensor. On left, a matrix is represented in a fibertree corresponding to CSC format, with a dense outer level and a sparse inner level. On right, a tensor is represented in a fibertree with two sparse outer levels, and a dense inner level. Note that the element levels in this case form the leaves of the tree.}
    \label{fig:levelsvsfibers}
\end{figure}

\subsection{Tensor Lifecycle, Declare, Freeze, Thaw, Unfurl}

Our simplified view of a level is enabled by our use of looplets to represent
the structure within each fiber. In fact, our level interface requires only
5 highly general operations, described below:

Our view of a level as a fiber allocator implies an allocation function
$assemble(tns, pos_{start}:pos_{stop})$, which allocates fibers at positions
$pos_{start}:pos_{stop}$ in the level. We don't specify a de-allocation function,
instead relying on initialization to reset the fiber if it needs to be reused.

\paragraph{$declare(lvl, init, dims...)$} Declares the level to hold subtensors of size $dims$ and an initial value of $init$. 
\paragraph{$declare(lvl, init, dims...)$} Declares the level to hold subtensors of size $dims$ and an initial value of $init$. 
\paragraph{$freeze(lvl, init, dims...)$} Declares the level to hold subtensors of size $dims$ and an initial value of $init$. 

\subsection{Core Level Language Primitives}
\begin{enumerate}
\item SparseList
\item SparseDict
\item ...
\end{enumerate}






\subsection{Wrapper Tensors}

\subsection{Scalars}

\subsubsection{Sparse Scalars}
\subsubsection{Early Break Scalars}


\section{The Finch Compiler}

\subsection{Dimensionalization}

\subsection{Concordization}

\subsection{Bounds Analysis}

\subsection{Performance Warnings}

\subsection{Wrapperization}

\subsection{Simplification and Algebraic Transformations}


\section{Evaluation}

\subsection{Data-Driven Performance Engineering}
\subsubsection{Sparse-Sparse Matrix Multiply}

Examples that demonstrate performance engineering in a datastructure-driven model

\subsubsection{SpMV}
Finch provides many flexible level formats to efficiently capture a variety of patterns in datasets, encompassing both the structure and type of data. More specifically, these formats can represent banded, triangular, run-length-encoded, blocked, hashed, and boolean data, as well as several other formats. Finch also provides the control flow necessary to manipulate the order in which data in these flexible level formats is read and written, enabling us to take advantage of multiple structural patterns concurrently—for example, we can exploit both sparsity and symmetry by using a sparse level format and restricting data reads to one triangle of a matrix. % I'm not sure if the right place for this is here - maybe should come earlier?

Structure in data arises both naturally, due to the chemical and physical properties of matter, and artificially via mathematical operations that induce particular patterns. Closely tailoring a storage format to a particular data pattern enables us to reduce the amount of stored values, make data accesses more efficient, and take advantage of spatial locality, resulting in more performant code.

The sparse matrix-vector multiplication kernel is a common operation in sparse linear algebra with many applications including conjugate gradients, graph algorithms, numerical analysis, and neural networks. The wide range of applications unsurprisingly results in a wide range of types of datasets making it an effective kernel to demonstrate the utility of having flexible data formats. In this case study, we highlight a few different Finch formats and the performance effects of conforming a dataset’s structure with its storage format, which Finch's datastructure-driven model enables us to do.

We compare Finch’s performance to that of TACO, SuiteSparseGraphBLAS, and Julia’s standard library.  We test using sparse matrices from a large selection of datasets spanning several previous papers: the datasets used by Vuduc et al. to test the OSKI interface [cite], Ahrens et al. to test a variable block row format partitioning strategy [cite], and Kjolstad et al. to test the TACO library [cite]. In addition, we also created several synthetic matrices containing bands or blocks of varying sizes as well as a permutation matrix to encapsulate a few additional use cases. The dense vector is randomly generated. We depict the performance of SpMV across the aforementioned tools and compare to the fastest Finch format for that particular dataset in Figures 1 and 2. 

\subsubsection{SparseList Format}
\subsubsection{SparseVBL Format}
The SparseVBL format is similar to the SparseList format, but contiguous fibers are stored together as blocks. Thus, it is most worthwhile to use when the matrix has a blocked structure—i.e. where there tend to be stretches of consecutive nonzero values. We found that the SparseVBL format performed the best for matrices with a clear block structure (list specific matrices) or an imperfect blocked structure (list specific matrices). 
\subsubsection{Pattern Format}
\subsubsection{Symmetric SpMV}
Finch enables us to exploit symmetry in the sparse matrix of the SpMV kernel by providing the capabilities to reuse memory reads and insert control flow logic to restrict iterations to either the lower or upper triangle of the sparse matrix. We can apply this strategy with any level format. Every symmetric matrix in the SparseList and SparseList-Pattern formats has better performance when we use a Finch SpMV program that takes advantage of this symmetry. However, the regular Finch SpMV program has better performance for symmetric matrices than the symmetric Finch SpMV program for the other more specialized formats, likely because we need in-order accesses to fully capitalize on the specialized storage. Symmetric SpMV with the SparseList level format in Finch results in an average of 1.3x speedup over TACO and symmetric SpMV with the SparseList-Pattern format in Finch results in an average speedup of 1.15x over TACO . Notably, there is a 1.9x speedup for the HB/saylr4 matrix. 
\subsubsection{4D Blocked SpMV}



%Here's a figure with spmv_performance_sorted_(faster_than_taco).png and spmv_performance_sorted_(slower_than_taco).png

\begin{figure}
    \begin{minipage}[t]{0.5\textwidth}
        \vspace{0pt} % Add this to ensure top alignment within minipage
        \includegraphics[width=\linewidth]{spmv_performance_sorted_(faster_than_taco).png}
    \end{minipage}%
    \begin{minipage}[t]{0.5\textwidth}
        \vspace{0pt} % Add this to ensure top alignment within minipage
        \includegraphics[width=\linewidth]{spmv_performance_sorted_(slower_than_taco).png}
    \end{minipage}
    \caption{Performance of SpMV across various tools.}
\end{figure}

\begin{figure}
    \includegraphics[width=\linewidth]{spmv_performance_grouped.png}
    \caption{Performance of SpMV by Finch format.}
\end{figure}


\subsection{Programming over flexible data}

\subsubsection{Image Morphology}

\begin{figure}
	\includegraphics[width=\linewidth]{erode4_speedup_over_opencv.png}
    \caption{Performance of Finch on erosion task (4 iterations).}
\end{figure}

\begin{figure}
	\includegraphics[width=\linewidth]{hist_speedup_over_opencv.png}
    \caption{Performance of Finch on masked histogram task.}
\end{figure}

\begin{figure}
	\includegraphics[width=\linewidth]{fill_speedup_over_opencv.png}
    \caption{Performance of Finch on flood fill task.}
\end{figure}

\subsubsection{Graph Analytics}
\begin{figure}
	\includegraphics[width=\linewidth]{bfs_speedup_over_graphs.jl.png}
	\includegraphics[width=\linewidth]{bellmanford_speedup_over_graphs.jl.png}
    \caption{Performance of graph apps across various tools.}
\end{figure}

\subsection{Implementing Numpy Array API in Finch}
\subsubsection{The Finch High-Level API (Needs a Name)}

\subsubsection{Finch Logic}

\subsubsection{Finch Interpreter}

\subsubsection{Lowering}
\subsubsection{Heuristic Optimization}

Find an example where fusing the python interface gives a big speedup over non-fused kernels.

%matmul, mttkrp, repeated ttm, triangle counting, multiple pointwise,
%in-place.
%dot((v^t .* u), w)) vs. 
%(v^t .* dot(u, w))

%%
%% The acknowledgments section is defined using the "acks" environment
%% (and NOT an unnumbered section). This ensures the proper
%% identification of the section in the article metadata, and the
%% consistent spelling of the heading.
\begin{acks}
    To Mateusz, Hameer, and Jaeyeon for their excellent programming contributions to the Finch codebase.
\end{acks}

%%
%% The next two lines define the bibliography style to be used, and
%% the bibliography file.
\bibliographystyle{ACM-Reference-Format}
\bibliography{bibliography.bib}


%%
%% If your work has an appendix, this is the place to put it.
\appendix

\end{document}
\endinput